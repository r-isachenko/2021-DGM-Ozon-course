\documentclass{beamer}
\usepackage[utf8]{inputenc}
\usepackage{graphicx, epsfig}
\usepackage{amsmath,mathrsfs,amsfonts,amssymb}
%\usepackage{subfig}
\usepackage{floatflt}
\usepackage{epic,ecltree}
\usepackage{mathtext}
\usepackage{fancybox}
\usepackage{fancyhdr}
\usepackage{multirow}
\usepackage{enumerate}
\usepackage{epstopdf}
\usepackage{multicol}
\usepackage{algorithm}
\usepackage[noend]{algorithmic}
\usepackage{tikz}
\usepackage{blindtext}
\usetheme{default}%{Singapore}%{Warsaw}%{Warsaw}%{Darmstadt}
\usecolortheme{default}
\setbeamerfont{title}{size=\Huge}
\setbeamertemplate{footline}[page number]{}
\setbeamerfont{title}{size=\Huge}
\beamertemplatenavigationsymbolsempty

\input{../utils/newcommands}

\newcommand{\createdgmtitle}[1]{\title[\hbox to 56mm{Deep Generative Models  \hfill\insertframenumber\,/\,\inserttotalframenumber}]
	{\vspace{1cm} \\ Deep Generative Models \\ Lecture #1 \\ \vspace{-0.5cm}}
	\author{Roman Isachenko \\ \vspace{-0.5cm}}
	\institute{\includegraphics[width=3cm]{../utils/ozonmasterslogo}
	\\Ozon Masters
	}
	\date{Spring, 2021}
}

\newcommand\myfootnote[1]{%
  \tikz[remember picture,overlay]
  \draw (current page.south west) +(1in + \oddsidemargin,0.5em)
  node[anchor=south west,inner sep=0pt]{\parbox{\textwidth}{%
      \rlap{\rule{10em}{0.4pt}}\raggedright\scriptsize#1}};}

\newcommand\myfootnotewithlink[2]{%
  \tikz[remember picture,overlay]
  \draw (current page.south west) +(1in + \oddsidemargin,0.5em)
  node[anchor=south west,inner sep=0pt]{\parbox{\textwidth}{%
      \rlap{\rule{10em}{0.4pt}}\raggedright\scriptsize\href{#1}{\textit{#2}}}};}
\createdgmtitle{6}
%--------------------------------------------------------------------------------
\begin{document}
%--------------------------------------------------------------------------------
\begin{frame}
%\thispagestyle{empty}
\titlepage
\end{frame}
%=======
\begin{frame}{Recap of previous lecture}
	\begin{figure}
		\includegraphics[width=0.7\linewidth]{figs/flows_how2.png}
	\end{figure}
	\vspace{-0.3cm}
		\begin{block}{Flow likelihood}
			\vspace{-0.3cm}
			\[
			\log p(\bx|\btheta) = \log p(f(\bx, \btheta)) + \log \left|\det \left( \frac{\partial f(\bx, \btheta)}{\partial \bx} \right) \right|
			\]
			\vspace{-0.5cm}
		\end{block}
		\begin{block}{What we want}
			\begin{itemize}
				\item Efficient computation of Jacobian $\frac{\partial f(\bx, \btheta)}{\partial \bx}$;
				\item Efficient sampling from the base distribution $p(\bz)$;
				\item Efficient inversion of $f(\bx, \btheta)$.
			\end{itemize}
			
		\end{block}
	\myfootnotewithlink{https://arxiv.org/abs/1605.08803}{Dinh L., Sohl-Dickstein J., Bengio S. Density estimation using Real NVP, 2016} 
\end{frame}
%=======
\begin{frame}{Recap of previous lecture}
	\vspace{-1.0cm}
	\begin{block}{Planar flow}
		\vspace{-0.5cm}
		\[
		g(\bz, \btheta) = \bz + \mathbf{u} \, h(\bw^T\bz + b).
		\]
		\vspace{-0.7cm}
	\end{block}
	\begin{block}{Sylvester flow}
		\vspace{-0.5cm}
		\[
			g(\bz, \btheta) = \bz + \bA \, h(\bB\bz + \mathbf{b}).
		\]
		\vspace{-0.7cm}
	\end{block}
	\begin{block}{NICE/RealNVP: Affine coupling law}
		\vspace{-0.3cm}
		\[
			\begin{cases} \bz_{1:d} = \bx_{1:d}; \\ \bz_{d:m} = \bx_{d:m} \odot \exp \left(c_1(\bx_{1:d}, \btheta)\right) + c_2(\bx_{1:d}, \btheta).\end{cases} 
		\]
		\vspace{-0.5cm}
	\end{block}
	\begin{block}{Glow: invertible 1x1 conv}
		\vspace{-0.3cm}
		\[
		\mathbf{W} = \mathbf{P}\mathbf{L}(\mathbf{U} + \text{diag}(\mathbf{s})).
		\]
	\end{block}
	\myfootnote{\href{https://arxiv.org/abs/1505.05770}{Rezende D. J., Mohamed S. Variational Inference with Normalizing Flows, 2015}\\
	\href{https://arxiv.org/abs/1803.05649}{Berg R. et al. Sylvester normalizing flows for variational inference, 2018} \\
	\href{https://arxiv.org/abs/1410.8516}{Dinh L., Krueger D., Bengio Y. NICE: Non-linear Independent Components Estimation, 2014} \\
	\href{https://arxiv.org/abs/1605.08803}{Dinh L., Sohl-Dickstein J., Bengio S. Density estimation using Real NVP, 2016} \\
	\href{https://arxiv.org/abs/1807.03039}{Kingma D. P., Dhariwal P. Glow: Generative Flow with Invertible 1x1 Convolutions, 2018}}
\end{frame}
%=======
\begin{frame}{Recap of previous lecture}
	\begin{block}{ELBO}
		\vspace{-0.3cm}
		\[
			p(\bx | \btheta) \geq \mathcal{L} (\bphi, \btheta)  = \mathbb{E}_{q(\bz | \bx, \bphi)} \log \frac{p(\bx, \bz | \btheta)}{q(\bz| \bx, \bphi)} \rightarrow \max_{\bphi, \btheta}.
		\]
		\vspace{-0.5cm}
	\end{block}
		\begin{itemize}
			\item Normal variational distribution $q(\bz | \bx, \bphi) = \mathcal{N}(\bz| \bmu_{\bphi}(\bx), \bsigma^2_{\bphi}(\bx))$ is poor (e.g. has only one mode). \\
			\item Flows models convert a simple base distribution to a compex one using invertible transformation with simple Jacobian. 
		\end{itemize}
	\begin{block}{Flow model in latent space}
		\vspace{-0.7cm}
		\[
			\log q_K(\bz_K | \bx, \bphi_*) = \log q(\bz_0 | \bx, \bphi) - \sum_{k=1}^K \log \left| \det \left( \frac{\partial g_k(\bz_{k - 1}, \bphi_k)}{\partial \bz_{k-1}} \right) \right|.
		\]
		\vspace{-0.5cm}
	\end{block}
	Let use $q_K(\bz_K | \bx, \bphi_*), \, \bphi_* = \{\bphi, \bphi_1, \dots, \bphi_K\}$ as a variational distribution. Here $\bphi$~-- encoder parameters, $\{\bphi_k\}_{k=1}^K$~-- flow parameters.
	
	\myfootnotewithlink{https://arxiv.org/abs/1505.05770}{Rezende D. J., Mohamed S. Variational Inference with Normalizing Flows, 2015} 
\end{frame}
%=======
\begin{frame}{Recap of previous lecture}
	\begin{block}{Variational distribution}
		\vspace{-0.6cm}
		\[
			\log q_K(\bz_K | \bx, \bphi_*) = \log q(\bz_0 | \bx, \bphi) - \sum_{k=1}^K \log \left| \det \left( \frac{\partial g_k(\bz_{k - 1}, \bphi_k)}{\partial \bz_{k-1}} \right) \right|.
		\]
		\vspace{-0.6cm}
	\end{block}
	\begin{block}{ELBO objective}
		\vspace{-0.7cm}
		\begin{align*}
			\mathcal{L} (\bphi, \btheta) 
			&= \mathbb{E}_{q(\bz_0 | \bx, \bphi)} \bigg[\log p(\bx, \bz_K | \btheta) -  \log q(\bz_0 | \bx, \bphi ) + \\ & \quad  + \sum_{k=1}^K \log \left| \det \left( \frac{\partial g_k(\bz_{k - 1}, \bphi_k)}{\partial \bz_{k-1}} \right) \right| \bigg].
		\end{align*}
		\vspace{-0.5cm}
	\end{block}
	\begin{itemize}
		\item Obtain samples $\bz_0$ from the encoder.
		\item Apply flow model $\bz_K = g(\bz_0, \{\bphi_k\}_{k = 1}^K)$.
		\item Compute likelihood for $\bz_K$ using the decoder, base distribution for $\bz_0$ and the Jacobian.
		\item We do not need inverse flow function, if we use flows in variational inference.
	\end{itemize}
	\myfootnotewithlink{https://arxiv.org/abs/1505.05770}{Rezende D. J., Mohamed S. Variational Inference with Normalizing Flows, 2015} 
\end{frame}
%=======
\begin{frame}{Dequantization}
	\begin{itemize}
		\item Images are discrete data, pixels lies in the [0, 255] integer domain (the model is $P(\bx | \btheta) = \text{Categorical}(\bpi(\btheta))$).
		\item Flow is a continuous model (it works with continuous data $\bx$).
	\end{itemize}
	Fitting a continuous density model to discrete data, could produce a degenerate solution with all probability mass on discrete values. \\
	How to convert discrete data distribution to the continuous one?
	
	\begin{minipage}[t]{0.5\columnwidth}
		\begin{block}{Uniform dequantization}
		\vspace{-0.5cm}
			\begin{align*}
				\bx &\sim \text{Categorical}(\bpi) \\
				 \bu &\sim U[0, 1]
			\end{align*}
			\[
			\by = \bx + \bu \sim \text{Continuous} 
			\]
		\end{block}
	\end{minipage}%
	\begin{minipage}[t]{0.5\columnwidth}
		\begin{figure}
			\centering
			\includegraphics[width=1.0\linewidth]{figs/uniform_dequantization.png}
		\end{figure}
	\end{minipage}
	\myfootnotewithlink{https://arxiv.org/abs/1511.01844}{Theis L., Oord A., Bethge M. A note on the evaluation of generative models. 2015}
\end{frame}
%=======
\begin{frame}{Uniform dequantization}
	\begin{block}{Statement}
		Fitting continious model $p(\by | \btheta)$ on uniformly dequantized data $\by = \bx + \bu, \, \bu \sim U[0, 1]$ is equivalent to maximization of a lower bound on the log-likelihood for a discrete model:
		\[
		P(\bx | \btheta) = \int_{U[0, 1]} p(\bx + \bu | \btheta) d \bu
		\]
		\vspace{-0.2cm} \\
		Thus, maximizing the log-likelihood of the continuous model on $\by$ cannot lead to the collapsing onto the discrete data (objective is bounded above by the log-likelihood of a discrete model).
	\end{block}
	\begin{block}{Proof}
		\vspace{-1cm}
		\begin{multline*}
			 \log P(\bx | \btheta) = \log \int_{U[0, 1]} p(\bx + \bu | \btheta) d \bu \geq \\ \geq \int_{U[0, 1]} \log p(\bx + \bu | \btheta) d \bu = \log p(\by | \btheta).
		\end{multline*}
	\end{block}
\end{frame}
%=======
\begin{frame}{Variational dequantization}
	\begin{minipage}[t]{0.5\columnwidth}
			\begin{figure}
				\centering
				\includegraphics[width=1.0\linewidth]{figs/uniform_dequantization.png}
			\end{figure}
	\end{minipage}%
	\begin{minipage}[t]{0.5\columnwidth}
		\begin{figure}
			\centering
			\includegraphics[width=1.0\linewidth]{figs/variational_dequantization.png}
		\end{figure}
	\end{minipage}
	\begin{itemize}
		\item $p(\by | \btheta)$ assign unifrom density to unit hypercubes $\bx + U[0, 1]$ (left fig).
		\item Neural network density models is a smooth function approximator (right fig).
		\item Smooth dequantization is more natural.
	\end{itemize}
	How to make the smooth dequantization? \\
\end{frame}
%=======
\begin{frame}{Flow++}
	\begin{block}{Variational dequantization}
		Introduce variational dequantization noise distribution $q(\bu | \bx)$ and treat it as an approximate posterior. 
	\end{block}
	\begin{block}{Variational lower bound}
		\vspace{-0.7cm}
		\begin{multline*}
		 \log P(\bx | \btheta) = \left[ \log \int q(\bu | \bx) \frac{p(\bx + \bu | \btheta)}{q(\bu | \bx)} d \bu \right] \geq \\ 
			\geq  \int q(\bu | \bx) \log \frac{p(\bx + \bu | \btheta)}{q(\bu | \bx)} d \bu = \mathcal{L}(q, \btheta).
		\end{multline*}
	\end{block}
	\vspace{-0.5cm}
	\begin{block}{Uniform dequantization bound}
		\vspace{-0.7cm}
		\begin{multline*}
		 \log P(\bx | \btheta) = \log \int_{U[0, 1]} p(\bx + \bu | \btheta) d \bu \geq \\ \geq \int_{U[0, 1]} \log p(\bx + \bu | \btheta) d \bu = \log p(\by | \btheta).
		\end{multline*}
	\end{block}
	\myfootnotewithlink{https://arxiv.org/abs/1902.00275}{Ho J. et al. Flow++: Improving Flow-Based Generative Models with Variational Dequantization and Architecture Design, 2019}
\end{frame}
%=======
\begin{frame}{Flow++}
	\begin{block}{Variational lower bound}
		\[
		\mathcal{L}(q, \btheta) = \int q(\bu | \bx) \log \frac{p(\bx + \bu | \btheta)}{q(\bu | \bx)} d \bu.
		\]
	\end{block}
	Let $\bu = h(\bepsilon, \bphi)$ is a flow model with base distribution $\bepsilon \sim p(\bepsilon) = \mathcal{N}(0, \mathbf{I})$:
	\vspace{-0.3cm}
	\[
		q(\bu | \bx) = p(h^{-1}(\bu, \bphi)) \cdot \left| \det \frac{\partial h^{-1}(\bu, \bphi)}{\partial \bu}\right|.
	\]
	\vspace{-0.3cm}
	Then
	\[
		\log P(\bx | \btheta) \geq \cL(\bphi, \btheta) = \int p(\bepsilon) \log \left( \frac{p(\bx + h(\bepsilon, \bphi) | \btheta)}{p(\bepsilon) \cdot \left| \det \frac{\partial h(\bepsilon, \bphi)}{\partial \bepsilon}\right|^{-1}} \right) d\bepsilon.
	\]
	\myfootnotewithlink{https://arxiv.org/abs/1902.00275}{Ho J. et al. Flow++: Improving Flow-Based Generative Models with Variational Dequantization and Architecture Design, 2019}
\end{frame}
%=======
\begin{frame}{Flow++}
	\begin{block}{Variational lower}
	\vspace{-0.3cm}
	\[
		\log P(\bx | \btheta) \geq \int p(\bepsilon)\log \left( \frac{p(\bx + h(\bepsilon, \bphi))}{p(\bepsilon) \cdot \left| \det \frac{\partial h(\bepsilon, \bphi)}{\partial \bepsilon}\right|^{-1}} \right) d\bepsilon.
	\]
	\end{block}
	\begin{itemize}
	\item If $p(\bx + \bu | \btheta)$ is also a flow model, it is straightforward to calculate stochastic gradient of this ELBO.
	
	\item Uniform dequantization is a special case of variational dequantization ($q(\bu | \bx) = U[0, 1]$).
	The gap between $\log P(\bx | \btheta)$ and the derived ELBO is 
	$ KL(q(\bu | \bx) || p(\bu | \bx))$.
	\item In the case of uniform dequantization the model unnaturally places uniform density over each hypercube $\bx + U[0, 1]$ due to inexpressive distribution $q$.
	\end{itemize}
	\myfootnotewithlink{https://arxiv.org/abs/1902.00275}{Ho J. et al. Flow++: Improving Flow-Based Generative Models with Variational Dequantization and Architecture Design, 2019}
\end{frame}
%=======
\begin{frame}{Flow++}
	\begin{figure}
		\centering
		\includegraphics[width=0.7\linewidth]{figs/flow++1.png}
	\end{figure}
	\vspace{-0.1cm}
	\begin{figure}
		\centering
		\includegraphics[width=0.8\linewidth]{figs/flow++2.png}
	\end{figure}
	\myfootnotewithlink{https://arxiv.org/abs/1902.00275}{Ho J. et al. Flow++: Improving Flow-Based Generative Models with Variational Dequantization and Architecture Design, 2019}
\end{frame}
%=======
\begin{frame}{Gaussian autoregressive model}
	Consider autoregressive model
	\[
		p(\bx | \btheta) = \prod_{i=1}^m p(x_i | \bx_{1:i - 1}, \btheta),
	\]
	with conditionals
	\[
	p(x_i | \bx_{1:i - 1}, \btheta) = \mathcal{N} \left(\hat{\mu}_i(\bx_{1:i-1}), \hat{\sigma}^2_i (\bx_{1:i-1})\right).
	\]
	\vspace{-0.5cm}
	\begin{block}{Sampling}
		\[
		x_i = \hat{\sigma}_i (\bx_{1:i-1}) \cdot z_i + \hat{\mu}_i(\bx_{1:i-1}), \quad z_i \sim \mathcal{N}(0, 1).
		\]
	\end{block}
	Sampling from autoregressive model is sequential. \\
	Note that we could interpret this sampling as a transformation $\bx = g(\bz, \btheta)$, where $\bz$ comes from base distribution $\mathcal{N}(0, 1)$.
\end{frame}
%=======
\begin{frame}{Gaussian autoregressive model}
	\begin{block}{Sampling}
		\vspace{-0.5cm}
		\[
		x_i = \hat{\sigma}_i (\bx_{1:i-1}) \cdot z_i + \hat{\mu}_i(\bx_{1:i-1}), \quad z_i \sim \mathcal{N}(0, 1).
		\]
		\vspace{-0.5cm}
	\end{block}
	\begin{block}{Jacobian}
		\vspace{-0.5cm}
		\[
		\log \left|\det \left( \frac{\partial f(\bx, \btheta)}{\partial \bx} \right) \right| = -\log \left|\det \left( \frac{\partial g(\bz, \btheta)}{\partial \bz} \right) \right| = - \sum_{i = 1}^m \log \hat{\sigma}_i (\bx_{1:i-1}).
		\]
		\vspace{-0.5cm}
	\end{block} 
	\begin{block}{Inverse transform}
		\vspace{-0.2cm}
		\[
		z_i = \left(x_i - \hat{\mu}_i(\bx_{1:i-1}) \right) \cdot \frac{1}{\hat{\sigma}_i (\bx_{1:i-1}) }.
		\]
	\end{block}
	We get an autoregressive model with tractable (triangular) Jacobian, which is easily invertible. It is a flow!
	\myfootnotewithlink{https://arxiv.org/abs/1606.04934}{Kingma D. P. et al. Improving Variational Inference with Inverse Autoregressive Flow, 2016} 
\end{frame}
%=======
\begin{frame}{Inverse autoregressive flow (IAF)}
	
	\begin{block}{Gaussian autoregressive model ($\bz \rightarrow \bx$)}
		\vspace{-0.2cm}
		\[
			x_i = \hat{\sigma}_i (\bx_{1:i-1}) \cdot z_i + \hat{\mu}_i(\bx_{1:i-1}).
		\]
		\[
			z_i = \left(x_i - \hat{\mu}_i(\bx_{1:i-1}) \right) \cdot \frac{1}{ \hat{\sigma}_i (\bx_{1:i-1})}.
		\]
		\vspace{-0.3cm}
	\end{block}
	This process is sequential. \\
	Let use the following reparametrization:
	$\bsigma = \frac{1}{\hat{\bsigma}}; \quad \bmu = - \frac{\hat{\bmu}}{\hat{\bsigma}}.$
	
	\begin{block}{Inverse transform ($\bx \rightarrow \bz$)}
		\vspace{-0.2cm}
		\[
			z_i = \sigma_i (\bx_{1:i-1}) \cdot x_i + \mu_i(\bx_{1:i-1}).
		\]
		\[
			x_i = \left( z_i - \mu_i(\bx_{1:i-1})\right) \cdot \frac{1}{\sigma_i (\bx_{1:i-1}) }.
		\]
		\vspace{-0.3cm}
	\end{block}
	This process is \textbf{not} sequential.
	\myfootnotewithlink{https://arxiv.org/abs/1606.04934}{Kingma D. P. et al. Improving Variational Inference with Inverse Autoregressive Flow, 2016} 
\end{frame}
%=======
\begin{frame}{Inverse autoregressive flow (IAF)}
	
	\begin{block}{Inverse transform ($\bx \rightarrow \bz$)}
		\vspace{-0.2cm}
		\[
			z_i = \sigma_i (\bx_{1:i-1}) \cdot x_i + \mu_i(\bx_{1:i-1}).
		\]
		\[
			x_i = \left( z_i - \mu_i(\bx_{1:i-1})\right) \cdot \frac{1}{\sigma_i (\bx_{1:i-1})}.
		\]
		\vspace{-0.3cm}
	\end{block}
	Inverse autoregressive flow use such inverted autoregressive model as a flow in VAE:
	\begin{align*}
		\bz_0 &= \bsigma(\bx) \cdot \bepsilon + \bmu(\bx), \quad \bepsilon \sim \mathcal{N}(0, 1); \quad  \sim q(\bz_0 | \bx, \bphi). \\
		\bz_k &= \bsigma_k(\bz_{k - 1}) \cdot \bz_{k-1} + \bmu_k(\bz_{k - 1}), \quad k \geq 1; \quad  \sim q_k(\bz_k | \bx, \bphi, \{\bphi_j\}_{j=1}^k).
	\end{align*}
	\myfootnotewithlink{https://arxiv.org/abs/1606.04934}{Kingma D. P. et al. Improving Variational Inference with Inverse Autoregressive Flow, 2016} 
\end{frame}
%=======
\begin{frame}{Flows}
	\begin{figure}
		\includegraphics[width=0.8\linewidth]{figs/flows_how2.png}
	\end{figure}
	\begin{itemize}	
		\item Inference mode in autoregressive flows is used for density estimation tasks.
		\item Generation mode in autoregressive flows (IAF) is used for stochastic variational inference to get a more flexible posterior distribution.
	\end{itemize}

	\myfootnotewithlink{https://arxiv.org/abs/1605.08803}{Dinh L., Sohl-Dickstein J., Bengio S. Density estimation using Real NVP, 2016} 
\end{frame}
%=======
\begin{frame}{Inverse autoregressive flow (IAF)}
	\begin{figure}
		\includegraphics[width=\linewidth]{figs/iaf2.png}
	\end{figure}
	\begin{figure}
		\includegraphics[width=\linewidth]{figs/iaf1.png}
	\end{figure}

	\myfootnotewithlink{https://arxiv.org/abs/1606.04934}{Kingma D. P. et al. Improving Variational Inference with Inverse Autoregressive Flow, 2016} 
\end{frame}
%=======
\begin{frame}{Inverse autoregressive flow (IAF)}

	\begin{minipage}[t]{0.65\columnwidth}
		\begin{block}{Gaussian autoregressive model}
			\[
			x_i = \hat{\sigma}_i (\bx_{1:i-1}) \cdot z_i + \hat{\mu}_i(\bx_{1:i-1}).
			\]
		\end{block}
	\end{minipage}%
	\begin{minipage}[t]{0.35\columnwidth}
		\begin{figure}[h]
			\centering
			\includegraphics[width=.9\linewidth]{figs/maf_iaf_explained_1.png}
		\end{figure}
	\end{minipage} \\
	
	\begin{minipage}[t]{0.65\columnwidth}
		\begin{block}{Inverse transform}
			\vspace{-0.5cm}
			\begin{align*}
				z_i &= (x_i - \hat{\mu}_i(\bx_{1:i-1})) \cdot \frac{1}{\hat{\sigma}_i (\bx_{1:i-1}) }; \\
				z_i &= \sigma_i (\bx_{1:i-1}) \cdot x_i + \mu_i(\bx_{1:i-1}).
			\end{align*}
			\vspace{-0.4cm}
		\end{block}
	\end{minipage}% 
	\begin{minipage}[t]{0.35\columnwidth}
		\begin{figure}[h]
			\centering
			\includegraphics[width=.9\linewidth]{figs/maf_iaf_explained_2.png}
		\end{figure}
	\end{minipage}\\
	\vspace{0.1cm}
	
	\begin{minipage}[t]{0.65\columnwidth}
		\begin{block}{Inverse autoregressive flow}
			\[
			x_i = \sigma_i (\bz_{1:i-1}) \cdot z_i + \mu_i(\bz_{1:i-1}).
			\]
		\end{block}
	\end{minipage}%
	\begin{minipage}[t]{0.35\columnwidth}
		\begin{figure}[h]
			\centering
			\includegraphics[width=.9\linewidth]{figs/maf_iaf_explained_3.png}
		\end{figure}
	\end{minipage}
	
	\myfootnotewithlink{https://blog.evjang.com/2018/01/nf2.html}{image credit: https://blog.evjang.com/2018/01/nf2.html}
\end{frame}
%=======
\begin{frame}{Masked autoregressive flow (MAF)}
	\begin{block}{Gaussian autoregressive model}
		\vspace{-0.5cm}
		\[
		p(\bx | \btheta) = \prod_{i=1}^m p(x_i | \bx_{1:i - 1}, \btheta) = \prod_{i=1}^m \mathcal{N} \left(x_i | \mu_i(\bx_{1:i-1}), \sigma^2_i (\bx_{1:i-1})\right).
		\]
		\vspace{-0.5cm}
	\end{block}
	We could use MADE (masked autoencoder) as conditional model. The sampling order could be crucial.
	\begin{figure}
		\includegraphics[width=\linewidth]{figs/maf1.png}
	\end{figure}
	Samples from the base distribution could be an indicator of how good the flow was fitted. \\
	\myfootnotewithlink{https://arxiv.org/abs/1705.07057}{Papamakarios G., Pavlakou T., Murray I. Masked Autoregressive Flow for Density Estimation, 2017} 
\end{frame}
%=======
\begin{frame}{Masked autoregressive flow (MAF)}
	\begin{block}{Gaussian autoregressive model}
		\vspace{-0.5cm}
		\[
		p(\bx | \btheta) = \prod_{i=1}^m p(x_i | \bx_{1:i - 1}, \btheta) = \prod_{i=1}^m \mathcal{N} \left(x_i | \mu_i(\bx_{1:i-1}), \sigma^2_i (\bx_{1:i-1})\right).
		\]
		\vspace{-0.5cm}
	\end{block}
	\begin{figure}
		\includegraphics[width=\linewidth]{figs/maf1.png}
	\end{figure}
	MAF is just a stacked MADE model.
	\myfootnotewithlink{https://arxiv.org/abs/1705.07057}{Papamakarios G., Pavlakou T., Murray I. Masked Autoregressive Flow for Density Estimation, 2017} 
\end{frame}
%=======
\begin{frame}{MAF vs IAF}
	\begin{block}{Sampling and inverse transform in MAF}
		\vspace{-0.2cm}
		\[
		x_i = \hat{\sigma}_i (\bx_{1:i-1}) \cdot z_i + \hat{\mu}_i(\bx_{1:i-1}).
		\]
		\[
		z_i = \left(x_i - \hat{\mu}_i(\bx_{1:i-1}) \right) \cdot \frac{1}{\hat{\sigma}_i (\bx_{1:i-1}) }.
		\]
		\vspace{-0.5cm}
		\begin{itemize}
			\item Sampling is slow (sequential).
			\item Density estimation is fast.
		\end{itemize}
	\end{block}
	\begin{block}{Sampling and inverse transform in IAF}
		\vspace{-0.2cm}
		\[
		x_i = \sigma_i (\bz_{1:i-1}) \cdot z_i + \mu_i(\bz_{1:i-1}).
		\]
		\[
		z_i = \left(x_i - \mu_i(\bz_{1:i-1}) \right) \cdot \frac{1}{\sigma_i (\bz_{1:i-1})}.
		\]
		\vspace{-0.3cm}
		\begin{itemize}
			\item Sampling is fast.
			\item Density estimation is slow (sequential).
		\end{itemize}
	\end{block}
	\myfootnotewithlink{https://arxiv.org/abs/1705.07057}{Papamakarios G., Pavlakou T., Murray I. Masked Autoregressive Flow for Density Estimation, 2017} 
\end{frame}
%=======
\begin{frame}{MAF vs IAF}
	\begin{block}{Theorem}
		Training a MAF with maximum likelihood corresponds to fitting an implicit IAF  with stochastic variational inference where the posterior is taken to be the base density $\pi(\bz)$:
		\[  
		\max_{\btheta} p(\bX | \btheta) \quad \Leftrightarrow \quad \min_{\btheta} KL\left(p(\bz | \btheta) || \pi(\bz)\right).
		\]
		\vspace{-0.5cm}
		\begin{itemize}
			\item $\pi(\bz)$ is a base distribution; $\pi(\bx)$ is a data distribution.
			\item $\bz = f(\bx, \btheta)$~-- MAF model; $\bx = g(\bz, \btheta)$~-- IAF model.
		\end{itemize}
	\end{block}
	\vspace{-0.3cm}
	\[
	\log p(\bz | \btheta) = \log \pi(g(\bz, \btheta)) + \log \left| \det \left( \frac{\partial g(\bz, \btheta)}{\partial \bz}\right) \right|
	\]
	\[
	\log p(\bx | \btheta) = \log \pi(f(\bx, \btheta)) + \log \left| \det \left( \frac{\partial f(\bx, \btheta)}{\partial \bx}\right) \right|
	\]
	\myfootnotewithlink{https://arxiv.org/abs/1705.07057}{Papamakarios G., Pavlakou T., Murray I. Masked Autoregressive Flow for Density Estimation, 2017} 
\end{frame}
%=======
\begin{frame}{MAF vs IAF}
	\begin{block}{Theorem}
		Training a MAF with maximum likelihood corresponds to fitting an implicit IAF  with stochastic variational inference where the posterior is taken to be the base density $\pi(\bz)$:
		\[  
		\max_{\btheta} p(\bX | \btheta) \quad \Leftrightarrow \quad \min_{\btheta} KL\left(p(\bz | \btheta) || \pi(\bz)\right).
		\]
		\vspace{-0.5cm}
	\end{block}
	\begin{block}{Proof}
		\vspace{-0.5cm}
		\begin{multline*}
			KL\left(p(\bz | \btheta) || \pi(\bz) \right) = \mathbb{E}_{p(\bz | \btheta)} \bigl[ \log p(\bz | \btheta) - \log \pi(\bz) \bigr] = \\ 
			= \mathbb{E}_{p(\bz | \btheta)} \left[ \log \pi(g(\bz, \btheta)) + \log \left| \det \left( \frac{\partial g(\bz, \btheta)}{\partial \bz}\right) \right| - \log \pi(\bz) \right] = \\
			= \mathbb{E}_{\pi(\bx)} \left[ \log \pi(\bx) - \log \left| \det \left( \frac{\partial f(\bx, \btheta)}{\partial \bx}\right) \right| - \log \pi(f(\bx, \btheta)) \right].
		\end{multline*}
	\end{block}
	\myfootnotewithlink{https://arxiv.org/abs/1705.07057}{Papamakarios G., Pavlakou T., Murray I. Masked Autoregressive Flow for Density Estimation, 2017} 
\end{frame}
%=======
\begin{frame}{MAF vs IAF}
	\begin{block}{Proof (continued)}
		\vspace{-0.5cm}
		\begin{multline*}
			KL\left(p(\bz | \btheta) || \pi(\bz) \right) = \\
			=  \mathbb{E}_{\pi(\bx)} \left[ \log \pi(\bx) - \log \left| \det \left( \frac{\partial f(\bx, \btheta)}{\partial \bx}\right) \right| - \log \pi(f(\bx, \btheta)) \right] = \\
			= \mathbb{E}_{\pi(\bx)} \bigl[ \log \pi(\bx) - \log p(\bx | \btheta) \bigr] = KL (\pi(\bx) || p(\bx | \btheta)).
		\end{multline*}
		\begin{align*}
			\argmin_{\btheta}  KL (\pi(\bx) || p(\bx | \btheta)) &= \argmin_{\btheta} \mathbb{E}_{\pi(\bx)} \left[ \log \pi(\bx) - \log p(\bx | \btheta) \right] \\
			&= \argmax_{\btheta} \mathbb{E}_{\pi(\bx)} \log p(\bx | \btheta)
		\end{align*}
		Unbiased estimator is MLE:
		\vspace{-0.2cm}
		\[
		\mathbb{E}_{\pi(\bx)} \log p(\bx | \btheta) = \sum_{i=1}^n \log p(\bx_i | \btheta).
		\]
		\vspace{-0.3cm}
	\end{block}
	\myfootnotewithlink{https://arxiv.org/abs/1705.07057}{Papamakarios G., Pavlakou T., Murray I. Masked Autoregressive Flow for Density Estimation, 2017} 
\end{frame}
%=======
\begin{frame}{MAF vs IAF vs RealNVP}
	\begin{block}{MAF}
		\vspace{-0.3cm}
		\[
		\bx = \hat{\bsigma} (\bx) \odot \bz + \hat{\bmu}(\bx).
		\]
		\vspace{-0.5cm}
		\begin{itemize}
			\item Calculating the density $p(\bx | \btheta)$ - 1 pass.
			\item Sampling - $m$ passes.
		\end{itemize}
	\end{block}
	\begin{block}{IAF}
		\vspace{-0.3cm}
		\[
		\bx = \bsigma (\bz) \odot \bz + \bmu(\bz).
		\]
		\vspace{-0.5cm}
		\begin{itemize}
			\item Calculating the density $p(\bx | \btheta)$ - $m$ passes.
			\item Sampling - 1 pass.
		\end{itemize}
	\end{block}
	\begin{block}{RealNVP}
		\vspace{-1cm}
		\begin{align*}
			\bx_{1:d} &= \bz_{1:d}; \\ \bx_{d:m} &= \bz_{d:m} \odot \exp \left(c_1(\bz_{1:d}, \btheta)\right) + c_2(\bx_{1:d}, \btheta).
		\end{align*}
		\vspace{-0.3cm}
	\end{block}
	\myfootnotewithlink{https://arxiv.org/abs/1705.07057}{Papamakarios G., Pavlakou T., Murray I. Masked Autoregressive Flow for Density Estimation, 2017} 
\end{frame}
%=======
\begin{frame}{MAF vs IAF vs RealNVP}
	\begin{block}{RealNVP}
		\vspace{-0.7cm}
		\begin{align*}
			\bx_{1:d} &= \bz_{1:d}; \\ \bx_{d:m} &= \bz_{d:m} \odot \exp \left(c_1(\bz_{1:d}, \btheta)\right) + c_2(\bx_{1:d}, \btheta).
		\end{align*}
		\vspace{-0.8cm}
	\end{block}
	\begin{itemize}
		\item Calculating the density $p(\bx | \btheta)$ - 1 pass.
		\item Sampling - 1 pass.
	\end{itemize}
	
	RealNVP is a special case of MAF and IAF:
	\begin{block}{MAF}
		\vspace{-0.5cm}
		\begin{equation*}
			\begin{cases}
				\hat{\mu}_i  = \hat{\sigma}_i = 0, \, i = 1, \dots, d; \\
				\hat{\mu}_i, \hat{\sigma}_i \text{ -- functions of } \bx_{1:d}, \, i = d+1, \dots, m.
			\end{cases}
		\end{equation*}
		\vspace{-0.3cm}
	\end{block}
	\begin{block}{IAF}
		\vspace{-0.3cm}
		\begin{equation*}
			\begin{cases}
				\mu_i  = \sigma_i = 0, \, i = 1, \dots, d; \\
				\mu_i, \sigma_i \text{ -- functions of } \bz_{1:d}, \, i = d+1, \dots, m.
			\end{cases}
		\end{equation*}
	\end{block}
	\myfootnotewithlink{https://arxiv.org/abs/1705.07057}{Papamakarios G., Pavlakou T., Murray I. Masked Autoregressive Flow for Density Estimation, 2017} 
\end{frame}
%=======
\begin{frame}{MAF/IAF pros and cons}
	\begin{block}{MAF}
		\begin{itemize}
			\item Sampling is slow.
			\item Likelihood evaluation is fast.
		\end{itemize}
	\end{block}
	\begin{block}{IAF}
		\begin{itemize}
			\item Sampling is fast.
			\item Likelihood evaluation is slow.
		\end{itemize}
	\end{block}
	\vspace{0.3cm}
	How to take the best of both worlds?
\end{frame}
%=======
\begin{frame}{WaveNet (2016)}
	Autoregressive model for raw audio waveforms generation
	\vspace{-0.2cm}
	\[
	p(\bx| \btheta) = \prod_{t=1}^T p(x_t|\bx_{1:t-1}, \btheta).
	\]
	\vspace{-0.2cm}
	The model uses causal dilated convolutions.
	\begin{figure}
		\centering
		\includegraphics[width=0.9\linewidth]{figs/wavenet2.png}
	\end{figure}
	\myfootnotewithlink{https://arxiv.org/abs/1609.03499}{Oord A. et al. Wavenet: A generative model for raw audio, 2016}
\end{frame}
%=======
\begin{frame}{Parallel WaveNet, 2017}
	\begin{block}{Previous WaveNet model}
		\begin{itemize}
			\item raw audio is high-dimensional (e.g. 16000 samples per second for 16kHz audio);
			\item WaveNet encodes 8-bit signal with 256-way categorical distribution.
		\end{itemize}
	\end{block}
	\begin{block}{Goal}
		\begin{itemize}
			\item improved fidelity (24kHz instead of 16kHz) $\rightarrow$ increase dilated convolution filter size from 2 to 3;
			\item 16-bit signals $\rightarrow$ mixture of logistics instead of categorical distribution.
		\end{itemize}
	\end{block}
	\myfootnotewithlink{https://arxiv.org/abs/1711.10433}{Oord A. et al. Parallel WaveNet: Fast High-Fidelity Speech Synthesis, 2017}
\end{frame}
%=======
\begin{frame}{Parallel WaveNet, 2017}
	\begin{block}{Probability density distillation}
		\begin{enumerate}
			\item Train usual WaveNet (MAF) via MLE (teacher network).
			\item Train IAF WaveNet model (student network), which attempts to match the probability of its own samples under the distribution learned by the teacher.
		\end{enumerate}
	\end{block}
	\begin{block}{Student objective}
		\[
		KL(p_s || p_t) = H(p_s, p_t) - H(p_s).
		\]
	\end{block}
	More than 1000x speed-up relative to original WaveNet!
	\myfootnotewithlink{https://arxiv.org/abs/1711.10433}{Oord A. et al. Parallel WaveNet: Fast High-Fidelity Speech Synthesis, 2017}
\end{frame}
%=======
\begin{frame}{Parallel WaveNet, 2017}
	\begin{figure}
		\centering
		\includegraphics[width=\linewidth]{figs/parallel_wavenet.png}
	\end{figure}

	\myfootnotewithlink{https://arxiv.org/abs/1711.10433}{Oord A. et al. Parallel WaveNet: Fast High-Fidelity Speech Synthesis, 2017}
\end{frame}
%=======
\begin{frame}{Summary}
	\begin{itemize}
		\item To apply continuous model to discrete distribution the standart practice is to dequantize data at first.
		\vfill
		\item Uniform dequantization is the simplest form of dequantization. Variational dequantization is more natural type that was proposed in Flow++ model.
		\vfill
		\item Gaussian autoregressive model is a special type of flow (RealNVP model is a special type of this autoregressive model).
		\vfill
		\item MAF is an example of such model which is suitable for density estimation tasks.
		\vfill
		\item  IAF used the inverse autoregressive transformation for variational inference task.
	\end{itemize}
\end{frame}
%=======
\end{document} 