\documentclass{beamer}
\usepackage[utf8]{inputenc}
\usepackage{graphicx, epsfig}
\usepackage{amsmath,mathrsfs,amsfonts,amssymb}
%\usepackage{subfig}
\usepackage{floatflt}
\usepackage{epic,ecltree}
\usepackage{mathtext}
\usepackage{fancybox}
\usepackage{fancyhdr}
\usepackage{multirow}
\usepackage{enumerate}
\usepackage{epstopdf}
\usepackage{multicol}
\usepackage{algorithm}
\usepackage[noend]{algorithmic}
\usepackage{tikz}
\usepackage{blindtext}
\usetheme{default}%{Singapore}%{Warsaw}%{Warsaw}%{Darmstadt}
\usecolortheme{default}
\setbeamerfont{title}{size=\Huge}
\setbeamertemplate{footline}[page number]{}
\setbeamerfont{title}{size=\Huge}
\beamertemplatenavigationsymbolsempty

\input{../utils/newcommands}

\newcommand{\createdgmtitle}[1]{\title[\hbox to 56mm{Deep Generative Models  \hfill\insertframenumber\,/\,\inserttotalframenumber}]
	{\vspace{1cm} \\ Deep Generative Models \\ Lecture #1 \\ \vspace{-0.5cm}}
	\author{Roman Isachenko \\ \vspace{-0.5cm}}
	\institute{\includegraphics[width=3cm]{../utils/ozonmasterslogo}
	\\Ozon Masters
	}
	\date{Spring, 2021}
}

\newcommand\myfootnote[1]{%
  \tikz[remember picture,overlay]
  \draw (current page.south west) +(1in + \oddsidemargin,0.5em)
  node[anchor=south west,inner sep=0pt]{\parbox{\textwidth}{%
      \rlap{\rule{10em}{0.4pt}}\raggedright\scriptsize#1}};}

\newcommand\myfootnotewithlink[2]{%
  \tikz[remember picture,overlay]
  \draw (current page.south west) +(1in + \oddsidemargin,0.5em)
  node[anchor=south west,inner sep=0pt]{\parbox{\textwidth}{%
      \rlap{\rule{10em}{0.4pt}}\raggedright\scriptsize\href{#1}{\textit{#2}}}};}
\createdgmtitle{7}
%--------------------------------------------------------------------------------
\begin{document}
%--------------------------------------------------------------------------------
\begin{frame}
%\thispagestyle{empty}
\titlepage
\end{frame}
%=======
\begin{frame}{VAE limitations}
	\begin{itemize}
		\item Poor variational posterior distribution (encoder)
		\[
		q(\bz | \bx, \bphi) = \mathcal{N}(\bz| \bmu_{\bphi}(\bx), \bsigma^2_{\bphi}(\bx)).
		\]
		\item Poor prior distribution
		\[
		p(\bz) = \mathcal{N}(0, \mathbf{I}).
		\]
		\item Poor probabilistic model (decoder)
		\[
		p(\bx | \bz, \btheta) = \mathcal{N}(\bx| \bmu_{\btheta}(\bz), \bsigma^2_{\btheta}(\bz)).
		\]
		\item Loose lower bound
		\[
		p(\bx | \btheta) - \mathcal{L}(q, \btheta) = (?).
		\]
	\end{itemize}
\end{frame}
%=======
\begin{frame}{VAE prior}
	\begin{block}{ELBO revisiting}
		\vspace{-0.3cm}
		{\footnotesize
			\[
			\mathcal{L}(q, \btheta) = \underbrace{\frac{1}{n} \sum_{i=1}^n \mathbb{E}_{q(\bz_i | \bx_i)} \log p(\bx_i | \bz_i, \btheta)}_{\text{Reconstruction loss}} - \underbrace{\left(\log n - \mathbb{E}_{q(\bz)} \mathbb{H} \left[ q(i | \bz) \right] \right)\vphantom{\sum_{i=1}}}_{0 \leq \text{Mutual info} \leq \log N } - \underbrace{KL(q(\bz) || p(\bz))\vphantom{\sum_{i=1}}}_{\text{Marginal KL}}
			\]}
	\end{block}
	
	How to choose the optimal $p(\bz)$?
	\begin{itemize}
		\item SG: $p(\bz) = \mathcal{N}(0, I)$ $\Rightarrow$ over-regularization;
		\vspace{0.1cm}
		\item MoG: $p(\bz | \blambda) = \frac{1}{K} \sum_{k=1}^K \mathcal{N}(\bmu_k, \bsigma_k^2)$ $\Rightarrow$ (*), (**);
		\vspace{0.1cm}
		\item $p(\bz) = q(\bz) = \frac{1}{n}\sum_{i=1}^n q(\bz | \bx_i)$ $\Rightarrow$ overfitting and highly expensive.
	\end{itemize}
	\myfootnote{\href{https://arxiv.org/abs/1611.02648}{(*) Dilokthanakul N. et al. Deep Unsupervised Clustering with Gaussian Mixture Variational Autoencoders, 2016} \\
	\href{https://pdfs.semanticscholar.org/f6fe/5e8e25994c188ba6a124462e2cc55f2c5a67.pdf}{(**) Nalisnick E., Hertel L., Smyth P. Approximate Inference for Deep Latent Gaussian Mixtures, 2016}}
	
\end{frame}
%=======
\begin{frame}{VampPrior}
	\begin{block}{Variational Mixture of posteriors}
		\[
		p(\bz | \blambda) = \frac{1}{K} \sum_{k=1}^K q(\bz | \bu_k),
		\]
		where $\blambda = \{\bu_1, \dots, \bu_K\}$ are trainable preudo-inputs.
	\end{block}
	\begin{itemize}
		\item Multimodal $\Rightarrow$ prevents over-regularization;.
		\item $K \ll n$ $\Rightarrow$ prevents from potential overfitting + less expensive to train.
		\item Pseudo-inputs are prior hyperparameters $\Rightarrow$ connection to the Empirical Bayes.
	\end{itemize}
	\myfootnotewithlink{https://arxiv.org/abs/1705.07120}{Tomczak J. M., Welling M. VAE with a VampPrior, 2017}
\end{frame}
%=======
\begin{frame}{VampPrior}
	Do we equally need the multimodal prior? \\
	\vspace{0.2cm}
	Is it beneficial to couple the prior with the variational posterior or MoG is enough?
	\begin{minipage}[t]{0.5\columnwidth}
		\begin{figure}[h]
			\centering
			\includegraphics[width=1.\linewidth]{figs/VampPrior_1.png}
		\end{figure}
	\end{minipage}%
	\begin{minipage}[t]{0.5\columnwidth}
		\begin{figure}[h]
			\centering
			\includegraphics[width=1.\linewidth]{figs/VampPrior_2.png}
		\end{figure}
	\end{minipage}
	\myfootnotewithlink{https://arxiv.org/abs/1705.07120}{Tomczak J. M., Welling M. VAE with a VampPrior, 2017}
\end{frame}
%=======
\begin{frame}{VampPrior}
	\vspace{0.1cm}
	\textbf{Top row:} images generated by PixelHVAE + VampPrior for chosen pseudo-input in the left top corner. \\
	\vspace{0.1cm}
	\textbf{Bottom row:} Images represent a subset of trained pseudo-inputs for different datasets.
	\begin{figure}[h]
		\centering
		\includegraphics[width=1.0\linewidth]{figs/VampPrior_4.png}
	\end{figure}
	\myfootnotewithlink{https://arxiv.org/abs/1705.07120}{Tomczak J. M., Welling M. VAE with a VampPrior, 2017}
\end{frame}
%=======
\begin{frame}{VAE limitations}
\begin{itemize}
	\item Poor variational posterior distribution (encoder)
	\[
	q(\bz | \bx, \bphi) = \mathcal{N}(\bz| \bmu_{\bphi}(\bx), \bsigma^2_{\bphi}(\bx)).
	\]
	\item Poor prior distribution
	\[
	p(\bz) = \mathcal{N}(0, \mathbf{I}).
	\]
	\item Poor probabilistic model (decoder)
	\[
	p(\bx | \bz, \btheta) = \mathcal{N}(\bx| \bmu_{\btheta}(\bz), \bsigma^2_{\btheta}(\bz)).
	\]
	\item Loose lower bound
	\[
	p(\bx | \btheta) - \mathcal{L}(q, \btheta) = (?).
	\]
\end{itemize}
\end{frame}
%=======
\begin{frame}{Posterior collapse: toy example}
	Let define latent variable model in the following way:
	\[
		p(\bx | \btheta) = \int p(\bx, \bz | \btheta) d \bz = \int p(\bx | \bz, \btheta) p(\bz) d \bz 
	\]
	\begin{itemize}
		\item prior distribution $p(\bz) = \mathcal{N}(\bz| 0, \mathbf{I})$;
		\item probabilistic model $p(\bx | \bz, \btheta) = \mathcal{N}(\bx | \bmu_{\btheta}(\bz), \bsigma_{\btheta}(\bz))$ (diagonal covariance);
		\item variational posterior $q(\bz | \bx, \bphi) =  \mathcal{N}(\bx | \bmu_{\bphi}(\bx), \bsigma_{\phi}(\bx))$  (diagonal covariance).
	\end{itemize}
	
	Let data distribution is $\pi(\bx) = \mathcal{N}(\bx | \bmu, \bSigma)$. Possible cases:
	\begin{itemize}
		\item covariance matrix $\bSigma$ is diagonal (univariate case);
		\item covariance matrix $\bSigma$ is \textbf{not} diagonal (multivariate case).
	\end{itemize}
	What is the difference?
\end{frame}
%=======
\begin{frame}{Posterior collapse: toy example}
	\begin{block}{Multivariate ($\bSigma$ is non-diagonal)}
		\vspace{-0.5cm}
		\begin{minipage}[t]{0.33\columnwidth}
			\begin{figure}[h]
				\centering
				\includegraphics[width=.8\linewidth]{figs/posterior_collapse_toy_1.png}
			\end{figure}
		\end{minipage}%
		\begin{minipage}[t]{0.33\columnwidth}
			\begin{figure}[h]
				\centering
				\includegraphics[width=0.75\linewidth]{figs/posterior_collapse_toy_3.png}
			\end{figure}
		\end{minipage}%
		\begin{minipage}[t]{0.33\columnwidth}
			\begin{figure}[h]
				\centering
				\includegraphics[width=.75\linewidth]{figs/posterior_collapse_toy_5.png}
			\end{figure}
		\end{minipage}
	The encoder uses latent variables to model data.
	\end{block}

	\begin{block}{Univariate ($\bSigma$ is diagonal)}
		\vspace{-0.5cm}
		\begin{minipage}[t]{0.33\columnwidth}
		\begin{figure}[h]
			\centering
			\includegraphics[width=.8\linewidth]{figs/posterior_collapse_toy_2.png}
		\end{figure}
		\end{minipage}%
		\begin{minipage}[t]{0.33\columnwidth}
		\begin{figure}[h]
			\centering
			\includegraphics[width=.75\linewidth]{figs/posterior_collapse_toy_4.png}
		\end{figure}
		\end{minipage}%
		\begin{minipage}[t]{0.33\columnwidth}
		\begin{figure}[h]
			\centering
			\includegraphics[width=.75\linewidth]{figs/posterior_collapse_toy_6.png}
		\end{figure}
		\end{minipage}
	Latent variables are not used, since the decoder could model the data without the encoder.
	\end{block}
\end{frame}
%=======
\begin{frame}{Posterior collapse}
	\begin{block}{Representation learning}
		"Identifies and disentangles the underlying causal factors of the data, so that it becomes easier to understand the data, to classify it, or to perform other tasks".
	\end{block}
	\[
		p(\bx | \btheta) = \int p(\bx, \bz | \btheta) d \bz = \int p(\bx | \bz, \btheta) p(\bz) d \bz 
	\]
	If the decoder model $p(\bx | \bz, \btheta)$ is powerful enough to model $p(\bx | \btheta)$ the latent variables $\bz$ becames irrelevant.
	
	\[
		\mathcal{L}(q, \btheta) = \frac{1}{n} \sum_{i=1}^n \left[ \mathbb{E}_{q(\bz_i | \bx_i)} \log p(\bx_i | \bz_i, \btheta) - KL(q(\bz_i | \bx_i) || p(\bz_i)) \right].
	\]
	Early in the training the approximate posterior $q(\bz|\bx)$ carries little information about $\bx$ and the model sets the posterior to the prior to avoid paying any cost $KL(q(\bz|\bx)||p(\bz))$.
\end{frame}
%=======
\begin{frame}{PixelVAE}
	\[
	    p(\bx | \btheta) = \int p(\bx, \bz | \btheta) d \bz = \int p(\bx | \bz, \btheta) p(\bz) d \bz 
	\]
	\begin{itemize}
		\item More powerful $p(\bx | \bz, \btheta)$ leads to more powerful generative model $p(\bx | \btheta)$.
		\item Too powerful $p(\bx | \bz, \btheta)$ could lead to posterior collapse, where variational posterior $q(\bz | \bx)$ will not carry any information about data and close to prior $p(\bz)$.
	\end{itemize}
	How to make the generative model $p(\bx | \bz, \btheta)$ more powerful?
	\begin{block}{Autoregressive decoder}
	\[
	    p(\bx | \bz , \btheta) = \prod_{i=1}^n p(x_i | \bx_{1:i - 1}, \bz , \btheta)
	\]
	\end{block}
	
	\myfootnotewithlink{https://arxiv.org/abs/1611.05013}{Gulrajani I. et al. PixelVAE: A Latent Variable Model for Natural Images, 2016}
\end{frame}
%=======
\begin{frame}{PixelVAE}
	VAE model with autoregressive PixelCNN decoder with few autoregressive layers. 
	\begin{itemize}
		\item Global structure is captured by latent variables.
		\item Local statistics are captured by limited receptive field autoregressive model.
	\end{itemize}
	\begin{figure}
	    \centering
	    \includegraphics[width=0.8\linewidth]{figs/PixelVAE_2.png}
	\end{figure}
		
	\myfootnotewithlink{https://arxiv.org/abs/1611.05013}{Gulrajani I. et al. PixelVAE: A Latent Variable Model for Natural Images, 2016}
\end{frame}
%=======
\begin{frame}{PixelVAE}
	\begin{minipage}[t]{0.5\columnwidth}
		MNIST
		\begin{figure}
			\centering
			\includegraphics[width=0.9\linewidth]{figs/PixelVAE_5.png}
		\end{figure}
	\end{minipage}%
	\begin{minipage}[t]{0.5\columnwidth}
		ImageNet 64x64
		\begin{figure}
			\centering
			\includegraphics[width=\linewidth]{figs/PixelVAE_4.png}
		\end{figure}
	\end{minipage}
	\begin{figure}
	    \centering
	    \includegraphics[width=0.5\linewidth]{figs/PixelVAE_3.png}
	\end{figure}
	
	\myfootnotewithlink{https://arxiv.org/abs/1611.05013}{Gulrajani I. et al. PixelVAE: A Latent Variable Model for Natural Images, 2016}
\end{frame}
%=======
\begin{frame}{Decoder weakening}
	\[
		p(\bx | \btheta) = \int p(\bx, \bz | \btheta) d \bz = \int p(\bx | \bz, \btheta) p(\bz) d \bz 
	\]
	Powerful decoder $p(\bx | \bz, \btheta)$ makes the model expressive, but posterior collapse is possible.
	
	PixelVAE model uses the autoregressive PixelCNN model with small number of layers to limit receptive field.
	
	How to force the model encode information about $\bx$ into $\bz$?
	\[
	    \mathcal{L}(q, \btheta) = \mathbb{E}_{q(\bz | \bx)} \log p(\bx | \bz, \btheta) - \beta \cdot KL (q(\bz | \bx) || p(\bz))
	\]
	What we get if $\beta = 1$ ($\beta = 0$)? \\
	
	\begin{block}{KL annealing}
		\begin{itemize}
		    \item Start training with $\beta = 0$.
		    \item Increase it until $\beta = 1$ during training process.
		\end{itemize}
	\end{block}
	\myfootnotewithlink{https://arxiv.org/abs/1511.06349}{Bowman S. R. et al. Generating Sentences from a Continuous Space, 2015}
\end{frame}
%=======
\begin{frame}{Decoder weakening}
	\begin{block}{Free bits}
	\begin{itemize}
	\item Divide the latent dimensions into the $K$ subsets.
	\item Ensure the use of less than $\lambda$ nats of information per subset $j$.
	\end{itemize}
	
	\[
	    \hat{\mathcal{L}}(q, \btheta) = \mathbb{E}_{q(\bZ | \bX)} \log p(\bX | \bZ, \btheta) - \sum_{j=1}^K \max(\lambda, KL (q(\bZ_j | \bX) || p(\bZ_j))).
	\]
	
	Increasing the latent information is advantageous for the reconstruction term. \\
	\vspace{0.2cm}
	This results in $KL (q(\bZ_j | \bx) || p(\bZ_j)) \geq \lambda$ for all $j$, in practice.
	\end{block}
	\myfootnotewithlink{https://arxiv.org/abs/1606.04934}{Kingma D. P. et al. Improving Variational Inference with Inverse Autoregressive Flow, 2016}
\end{frame}
%=======
\begin{frame}{Variational Lossy AutoEncoder, 2016}
	\begin{block}{Lossy code via explicit information placement}
	\[
	    p(\bx | \bz, \btheta) = \prod_{i=1}^m p(x_i | \bz, \bx_{\text{WindowAround}(i)}, \btheta).
	\]
	\begin{itemize}
	    \item WindowAround($i$) restricts the receptive field (it forbids to represent arbitrarily complex distribution over $\bx$ without dependence on $\bz$). 
	    \item Local statistics of 2D images (texture) will be modeled by a small local window.
	    \item Global structural information (shapes) is long-range dependency that can only be communicated through latent code $\bz$. 
	\end{itemize}
	\end{block}
	\myfootnotewithlink{https://arxiv.org/abs/1611.02731}{Chen X. et al. Variational Lossy Autoencoder, 2016}
\end{frame}
%=======
\begin{frame}{Variational Lossy AutoEncoder, 2016}
	
	\begin{block}{ELBO revisiting}
		\vspace{-0.3cm}
		{\footnotesize
			\[
			\mathcal{L}(q, \btheta) = \underbrace{\frac{1}{n} \sum_{i=1}^n \mathbb{E}_{q(\bz_i | \bx_i)} \log p(\bx_i | \bz_i, \btheta)}_{\text{Reconstruction loss}} - \underbrace{\left(\log n - \mathbb{E}_{q(\bz)} \mathbb{H} \left[ q(i | \bz) \right] \right)\vphantom{\sum_{i=1}}}_{0 \leq \text{Mutual info} \leq \log N } - \underbrace{KL(q(\bz) || p(\bz))\vphantom{\sum_{i=1}}}_{\text{Marginal KL}}
			\]}
	\end{block}
	\vspace{-0.5cm}
	\begin{block}{VampPrior}
		\vspace{-0.5cm}
		\[
			p(\bz | \blambda) = \frac{1}{K} \sum_{k=1}^K q(\bz | \bu_k),
		\]
	where $\bu_1, \dots \bu_K$ is trainable preudo-inputs.
	\end{block}
	\begin{block}{Autoregressive flow prior}
		\vspace{-0.5cm}
		\[
			\log p(\bz | \blambda) = \log p(\bepsilon) + \log \det \left | \frac{d \bepsilon}{d\bz}\right|
		\]
		\[
			\bz = g(\bepsilon, \blambda) = f^{-1}(\bepsilon, \blambda) 
		\]
	\end{block}
	\myfootnotewithlink{https://arxiv.org/abs/1611.02731}{Chen X. et al. Variational Lossy Autoencoder, 2016}
\end{frame}
%=======
\begin{frame}{Variational Lossy AutoEncoder, 2016}
	\begin{block}{Theorem}
	VAE with the AF prior for latent code $\bz$ is equivalent to using the IAF posterior for latent code $\bepsilon$.
	\end{block}
	\begin{block}{Proof}
	\vspace{-0.5cm}
	{\footnotesize
	\begin{align*}
	\mathcal{L}(q, \btheta) &= \mathbb{E}_{q(\bz | \bx)} \left[ \log p(\bx | \bz, \btheta) +  \log p(\bz, \blambda) - q(\bz | \bx) \right] \\
	&= \mathbb{E}_{\bz \sim q(\bz | \bx)} \Bigl[ \log p(\bx | \bz, \btheta) + \underbrace{ \Bigl( \log p(f(\bz, \blambda)) + \log \left| \det \frac{\partial f(\bz, \blambda)}{\partial \bz} \right| \Bigr) }_{\text{AF prior}} - q(\bz | \bx) \Bigr] \\
	&= \mathbb{E}_{\bz \sim q(\bz | \bx)} \Bigl[ \log p(\bx | \bz, \btheta) +  \log p(f(\bz, \blambda)) - \underbrace{ \Bigl( q(\bz | \bx) - \log \left| \det \frac{\partial f(\bz, \blambda)}{\partial \bz} \right| \Bigr) }_{\text{IAF posterior}} \Bigr]
	\end{align*}
	}
	\end{block}
	\myfootnotewithlink{https://arxiv.org/abs/1611.02731}{Chen X. et al. Variational Lossy Autoencoder, 2016}
\end{frame}
%=======
\begin{frame}{Variational Lossy AutoEncoder, 2016}
	\begin{block}{Autoregressive flow prior}
		{\footnotesize
		\begin{align*}
			\mathcal{L}(q, \btheta) &= \mathbb{E}_{\bz \sim q(\bz | \bx)} \Bigl[ \log p(\bx | \bz, \btheta) + \underbrace{ \Bigl( \log p(f(\bz, \blambda)) + \log \left| \det \frac{\partial f(\bz, \blambda)}{\partial \bz} \right| \Bigr) }_{\text{AF prior}} - q(\bz | \bx) \Bigr] \\
			&= \mathbb{E}_{\bz \sim q(\bz | \bx)} \Bigl[ \log p(\bx | \bz, \btheta) +  \log p(f(\bz, \blambda)) - \underbrace{ \Bigl( q(\bz | \bx) - \log \left| \det \frac{\partial f(\bz, \blambda)}{\partial \bz} \right| \Bigr) }_{\text{IAF posterior}} \Bigr]
		\end{align*}
		}
	\end{block}
	\begin{itemize}
		\item IAF posterior decoder path: $p(\bx|\bz, \btheta)$, $\bz \sim p(\bz)$.
		\item AF prior decoder path: $p(\bx|\bz, \btheta)$, $\bz = g(\bepsilon, \blambda)$, $\bepsilon \sim p(\bepsilon)$. 
	\end{itemize}
	The AF prior and the IAF posterior have the same computation cost, so using the AF prior makes the model more expressive at no training time cost.

	\myfootnotewithlink{https://arxiv.org/abs/1611.02731}{Chen X. et al. Variational Lossy Autoencoder, 2016}
\end{frame}
%=======
\begin{frame}{Variational Lossy AutoEncoder, 2016}
\begin{itemize}
    \item Can VLAE learn lossy codes that encode global statistics?
    \item Does using AF priors improves upon using IAF posteriors as predicted by theory?
    \item Does using autoregressive decoding distributions improve density estimation performance?
\end{itemize}
	\begin{minipage}[t]{0.5\columnwidth}
	\vspace{1cm}
		MNIST
		\begin{figure}[h]
			\centering
			\includegraphics[width=1.\linewidth]{figs/VLAE_1.png}
		\end{figure}
	\end{minipage}%
	\begin{minipage}[t]{0.5\columnwidth}
		CIFAR10
		\begin{figure}[h]
			\centering
			\includegraphics[width=1.\linewidth]{figs/VLAE_2.png}
		\end{figure}
	\end{minipage}

	\myfootnotewithlink{https://arxiv.org/abs/1611.02731}{Chen X. et al. Variational Lossy Autoencoder, 2016}
\end{frame}
%=======
\begin{frame}{VAE limitations}
	\begin{itemize}
		\item Poor variational posterior distribution (encoder)
		\[
		q(\bz | \bx, \bphi) = \mathcal{N}(\bz| \bmu_{\bphi}(\bx), \bsigma^2_{\bphi}(\bx)).
		\]
		\item Poor prior distribution
		\[
		p(\bz) = \mathcal{N}(0, \mathbf{I}).
		\]
		\item Poor probabilistic model (decoder)
		\[
		p(\bx | \bz, \btheta) = \mathcal{N}(\bx| \bmu_{\btheta}(\bz), \bsigma^2_{\btheta}(\bz)).
		\]
		\item Loose lower bound
		\[
		p(\bx | \btheta) - \mathcal{L}(q, \btheta) = (?).
		\]
	\end{itemize}
\end{frame}
%=======
\begin{frame}{Importance Sampling}
	\begin{block}{Generative model}
		\vspace{-0.5cm}
		\begin{align*}
			p(\bx | \btheta) &= \int p(\bx, \bz | \btheta) d\bz = \int \left[\frac{p(\bx, \bz | \btheta)}{q(\bz | \bx)} \right] q(\bz | \bx) d\bz \\
			&= \int f(\bx, \bz) q(\bz | \bx) d\bz = \mathbb{E}_{\bz \sim q(\bz | \bx)} f(\bx, \bz)
		\end{align*}
	\end{block}
	Here $f(\bx, \bz) = \frac{p(\bx, \bz | \btheta)}{q(\bz | \bx)}$.
	\begin{block}{ELBO}
		\vspace{-0.5cm}
		\begin{multline*}
			\log p(\bx | \btheta) = \log \mathbb{E}_{\bz \sim q(\bz | \bx)} f(\bx, \bz)
			\geq \mathbb{E}_{\bz \sim q(\bz | \bx)} \log f(\bx, \bz) = \\
			= \mathbb{E}_{\bz \sim q(\bz | \bx)} \log \frac{p(\bx, \bz | \btheta)}{q(\bz | \bx)} = \mathcal{L}(q, \btheta).
		\end{multline*}
	\end{block}
	Could we choose better $f(\bx, \bz)$? 
\end{frame}
%=======
\begin{frame}{IWAE}
	Let define
	\[
	f(\bx, \bz_1, \dots, \bz_K) = \frac{1}{K} \sum_{k=1}^K \frac{p(\bx, \bz_k | \btheta)}{q(\bz_k | \bx)}
	\]
	\[
		\mathbb{E}_{\bz_1, \dots, \bz_K \sim q(\bz | \bx)} f(\bx, \bz_1, \dots, \bz_K) = p(\bx | \btheta)
	\]
	\vspace{-0.3cm}
	\begin{block}{ELBO}
		\vspace{-0.5cm}
		\begin{multline*}
			\log p(\bx | \btheta) = \log \mathbb{E}_{\bz_1, \dots, \bz_K \sim q(\bz | \bx)} f(\bx, \bz, \dots, \bz_K) \geq \\
			\geq \mathbb{E}_{\bz_1, \dots, \bz_K \sim q(\bz | \bx)} \log f(\bx, \bz, \dots, \bz_K) = \\
			= \mathbb{E}_{\bz_1, \dots, \bz_K \sim q(\bz | \bx)} \log \left[\frac{1}{K} \sum_{k=1}^K\frac{p(\bx, \bz_k | \btheta)}{q(\bz_k | \bx)} \right] = \mathcal{L}_K(q, \btheta).
		\end{multline*}
	\end{block}
	\myfootnotewithlink{https://arxiv.org/abs/1509.00519}{Burda Y., Grosse R., Salakhutdinov R. Importance Weighted Autoencoders, 2015}
\end{frame}
%=======
\begin{frame}{IWAE}
	\begin{block}{VAE objective}
		\vspace{-0.2cm}
		\[
		\log p(\bx | \btheta) \geq \mathcal{L} (q, \btheta)  = \mathbb{E}_{q(\bz | \bx)} \log \frac{p(\bx, \bz | \btheta)}{q(\bz| \bx)} \rightarrow \max_{\bphi, \btheta}
		\]
		\[
		\mathcal{L} (q, \btheta)  = \mathbb{E}_{\bz_1, \dots, \bz_K \sim q(\bz | \bx)} \left( \frac{1}{K} \sum_{k=1}^K \log \frac{p(\bx, \bz_k | \btheta)}{q(\bz_k| \bx)} \right) \rightarrow \max_{q, \btheta}.
		\]
		\vspace{-0.2cm}
	\end{block}
	\begin{block}{IWAE objective}
		\vspace{-0.2cm}
		\[
		\mathcal{L}_K (q, \btheta)  = \mathbb{E}_{\bz_1, \dots, \bz_K \sim q(\bz | \bx)} \log \left( \frac{1}{K}\sum_{k=1}^K\frac{p(\bx, \bz_k | \btheta)}{q(\bz_k| \bx)} \right) \rightarrow \max_{q, \btheta}.
		\]
	\end{block}
	If $K=1$, these objectives coincide.

	\myfootnotewithlink{https://arxiv.org/abs/1509.00519}{Burda Y., Grosse R., Salakhutdinov R. Importance Weighted Autoencoders, 2015}
\end{frame}
%=======
\begin{frame}{IWAE}
	\begin{block}{Theorem}
		\begin{enumerate}
			\item $\log p(\bx | \btheta) \geq \mathcal{L}_K (q, \btheta) \geq \mathcal{L}_M (q, \btheta), \quad \text{for } K \geq M$;
			\item $\log p(\bx | \btheta) = \lim_{K \rightarrow \infty} \mathcal{L}_K (q, \btheta)$ if $\frac{p(\bx, \bz | \btheta)}{q(\bz | \bx)}$ is bounded.
		\end{enumerate}
		\vspace{-0.2cm}
	\end{block}
	\begin{block}{Proof of 1.}
		{ \footnotesize
			\vspace{-0.5cm}
			\begin{align*}
				\mathcal{L}_K (q, \btheta) &= \mathbb{E}_{\bz_1, \dots, \bz_K} \log \left( \frac{1}{K}\sum_{k=1}^K\frac{p(\bx, \bz_k | \btheta)}{q(\bz_k| \bx)} \right) = \\
				&= \mathbb{E}_{\bz_1, \dots, \bz_K} \log \mathbb{E}_{k_1, \dots, k_M} \left( \frac{1}{M}\sum_{m=1}^M\frac{p(\bx, \bz_{k_m} | \btheta)}{q(\bz_{k_m}| \bx)} \right) \geq \\
				&\geq \mathbb{E}_{\bz_1, \dots, \bz_K} \mathbb{E}_{k_1, \dots, k_m} \log \left( \frac{1}{M}\sum_{m=1}^M\frac{p(\bx, \bz_{k_m} | \btheta)}{q(\bz_{k_m}| \bx)} \right) = \\
				&= \mathbb{E}_{\bz_1, \dots, \bz_M} \log \left( \frac{1}{M}\sum_{m=1}^M\frac{p(\bx, \bz_m | \btheta)}{q(\bz_m| \bx)} \right) = \mathcal{L}_M (q, \btheta)
			\end{align*}
		}
	\end{block}
	
	\myfootnotewithlink{https://arxiv.org/abs/1509.00519}{Burda Y., Grosse R., Salakhutdinov R. Importance Weighted Autoencoders, 2015}
\end{frame}
%=======
\begin{frame}{IWAE}
	\begin{block}{Theorem}
		\begin{enumerate}
			\item $\log p(\bx | \btheta) \geq \mathcal{L}_K (q, \btheta) \geq \mathcal{L}_M (q, \btheta), \quad \text{for} K \geq M$;
			\item $\log p(\bx | \btheta) = \lim_{K \rightarrow \infty} \mathcal{L}_K (q, \btheta)$ if $\frac{p(\bx, \bz | \btheta)}{q(\bz | \bx)}$ is bounded.
		\end{enumerate}
		\vspace{-0.2cm}
	\end{block}
	\begin{block}{Proof of 2.}
		\vspace{0.2cm}
		Consider r.v. $\xi_K = \frac{1}{K}\sum_{k=1}^K \frac{p(\bx, \bz_k | \btheta)}{q(\bz_k | \bx)}$. \\
		\vspace{0.2cm}
		If summands are bounded, then (from the strong law of large numbers)
		\[
		\xi_K \xrightarrow[K \rightarrow \infty]{a.s.} \mathbb{E}_{q(\bz | \bx)} \frac{p(\bx, \bz | \btheta)}{q(\bz | \bx)} = p(\bx | \btheta).
		\]
		Hence $\mathcal{L}_K (q, \btheta) = \mathbb{E} \log \xi_K$ converges to $\log p(\bx | \btheta)$ as $K \rightarrow \infty$.
	\end{block}

	\myfootnotewithlink{https://arxiv.org/abs/1509.00519}{Burda Y., Grosse R., Salakhutdinov R. Importance Weighted Autoencoders, 2015}
\end{frame}
%=======
\begin{frame}{IWAE}
	\[
	\log p(\bx | \btheta) \geq \mathcal{L}_K(q, \btheta) \geq \mathcal{L}(q, \btheta)
	\]
	If $K > 1$ the bound could be tighter.
	\begin{align*}
		\mathcal{L} (q, \btheta) &= \mathbb{E}_{q(\bz | \bx)} \log \frac{p(\bx, \bz | \btheta)}{q(\bz| \bx)}; \\
		\mathcal{L}_K (q, \btheta) &= \mathbb{E}_{\bz_1, \dots, \bz_K \sim q(\bz | \bx)} \log \left( \frac{1}{K}\sum_{k=1}^K\frac{p(\bx, \bz_k | \btheta)}{q(\bz_k| \bx)} \right).
	\end{align*}
	\vspace{-0.2cm}
	\begin{itemize}
		\item $\mathcal{L}_1(q, \btheta) = \mathcal{L}(q, \btheta)$;
		\item $\mathcal{L}_{\infty}(q, \btheta) = \log p(\bx | \btheta)$.
	\end{itemize}
	\vspace{0.2cm}
	Which $q(\bz | \bx)$ gives $\mathcal{L}(q, \btheta) = \log p(\bx | \btheta)$? \\
	\vspace{0.2cm}
	Which $q(\bz | \bx)$ gives $\mathcal{L}(q, \btheta) = \mathcal{L}_K(q, \btheta)$?

	\myfootnotewithlink{https://arxiv.org/abs/1509.00519}{Burda Y., Grosse R., Salakhutdinov R. Importance Weighted Autoencoders, 2015}
\end{frame}
%=======
\begin{frame}{IWAE}
	\begin{block}{Theorem}
		The VAE objective is equal to IWAE objective 
		\[
		\mathcal{L}(q_{EW}, \btheta) = \mathcal{L}_K(q, \btheta)
		\]
		for the following variational distribution
		\[
		q_{EW}(\bz | \bx) = \mathbb{E}_{\bz_2, \dots, \bz_K \sim q(\bz | \bx)} q_{IW}(\bz | \bx, \bz_{2:K}),
		\]
		where \[
		q_{IW}(\bz | \bx, \bz_{2:K}) = \frac{\frac{p(\bx, \bz)}{q(\bz | \bx)}}{\frac{1}{K} \sum_{k=1}^K \frac{p(\bx, \bz_k)}{q(\bz_k | \bx)}} q(\bz | \bx) = \frac{p(\bx, \bz)}{\frac{1}{K}\left( \frac{p(\bx, \bz)}{q(\bz | \bx)} + \sum_{k=2}^K \frac{p(\bx, \bz_k)}{q(\bz_k | \bx)}\right)}.
		\]
	\end{block}
	
	\myfootnotewithlink{https://arxiv.org/abs/1704.02916}{Cremer C., Morris Q., Duvenaud D. Reinterpreting Importance-Weighted Autoencoders, 2017, 2017}
\end{frame}
%=======
\begin{frame}{IWAE}
	\begin{figure}
		\centering
		\includegraphics[width=\linewidth]{figs/IWAE_1.png}
	\end{figure}
	\begin{figure}
		\centering
		\includegraphics[width=\linewidth]{figs/IWAE_2.png}
	\end{figure}

	\myfootnotewithlink{https://arxiv.org/abs/1704.02916}{Cremer C., Morris Q., Duvenaud D. Reinterpreting Importance-Weighted Autoencoders, 2017, 2017}
\end{frame}
%=======
\begin{frame}{Summary}
	\begin{itemize}
		\item The ELBO surgery reveals insights about a prior distribution in VAE. The optimal prior is the aggregated posterior.
		\item VampPrior proposes to use a variational mixture of posteriors as the prior to approximate the aggregated posterior.
		\item More powerful decoder in VAE leads to more expressive generative model. However, too expressive decoder could lead to the posterior collapse.
		\item The decoder weakening is a set of techniques to avoid the posterior collapse.
		\item The autoregressive flows could be used as the prior. This is equivalent to the use of the IAF posterior. 
		\item The importance sampling could get the tighter lower bound to the likelihood.
	\end{itemize}
\end{frame}
%=======
\begin{frame}{IWAE}
\[
\frac{a_1 + \dots + a_K}{K} = \mathbb{E}_{i_1, \dots, i_M} \frac{a_{i_1} + \dots + a_{i_M}}{M}, \quad i_1, \dots, i_M \sim U[1, K]
\]
\end{frame}

\end{document} 