\documentclass{beamer}
\usepackage[utf8]{inputenc}
\usepackage{graphicx, epsfig}
\usepackage{amsmath,mathrsfs,amsfonts,amssymb}
%\usepackage{subfig}
\usepackage{floatflt}
\usepackage{epic,ecltree}
\usepackage{mathtext}
\usepackage{fancybox}
\usepackage{fancyhdr}
\usepackage{multirow}
\usepackage{enumerate}
\usepackage{epstopdf}
\usepackage{multicol}
\usepackage{algorithm}
\usepackage[noend]{algorithmic}
\usepackage{tikz}
\usepackage{blindtext}
\usetheme{default}%{Singapore}%{Warsaw}%{Warsaw}%{Darmstadt}
\usecolortheme{default}
\setbeamerfont{title}{size=\Huge}
\setbeamertemplate{footline}[page number]{}
\setbeamerfont{title}{size=\Huge}
\beamertemplatenavigationsymbolsempty

\input{../utils/newcommands}

\newcommand{\createdgmtitle}[1]{\title[\hbox to 56mm{Deep Generative Models  \hfill\insertframenumber\,/\,\inserttotalframenumber}]
	{\vspace{1cm} \\ Deep Generative Models \\ Lecture #1 \\ \vspace{-0.5cm}}
	\author{Roman Isachenko \\ \vspace{-0.5cm}}
	\institute{\includegraphics[width=3cm]{../utils/ozonmasterslogo}
	\\Ozon Masters
	}
	\date{Spring, 2021}
}

\newcommand\myfootnote[1]{%
  \tikz[remember picture,overlay]
  \draw (current page.south west) +(1in + \oddsidemargin,0.5em)
  node[anchor=south west,inner sep=0pt]{\parbox{\textwidth}{%
      \rlap{\rule{10em}{0.4pt}}\raggedright\scriptsize#1}};}

\newcommand\myfootnotewithlink[2]{%
  \tikz[remember picture,overlay]
  \draw (current page.south west) +(1in + \oddsidemargin,0.5em)
  node[anchor=south west,inner sep=0pt]{\parbox{\textwidth}{%
      \rlap{\rule{10em}{0.4pt}}\raggedright\scriptsize\href{#1}{\textit{#2}}}};}
\createdgmtitle{11}
%--------------------------------------------------------------------------------
\begin{document}
%--------------------------------------------------------------------------------
\begin{frame}
%\thispagestyle{empty}
\titlepage
\end{frame}
%=======
\begin{frame}{Evolution of GANs}
	\begin{figure}
		\centering
		\includegraphics[width=\linewidth]{figs/gan_evolution}
	\end{figure}
	\begin{itemize}
		\item \textbf{Vanilla GAN} \href{https://arxiv.org/abs/1406.2661}{https://arxiv.org/abs/1406.2661}
		\item \textbf{DCGAN} \href{https://arxiv.org/abs/1511.06434}{https://arxiv.org/abs/1511.06434}
		\item \textbf{CoGAN} \href{https://arxiv.org/abs/1606.07536}{https://arxiv.org/abs/1606.07536}
		\item \textbf{ProGAN} \href{https://arxiv.org/abs/1710.10196}{https://arxiv.org/abs/1710.10196} 
		\item \textbf{StyleGAN} \href{https://arxiv.org/abs/1812.04948}{https://arxiv.org/abs/1812.04948}
	\end{itemize}
\end{frame}
%=======
\begin{frame}{Self-Attention GAN}
	\begin{itemize}
		\item Convolutional layers process the information in a local neighborhood.
		\item Using convolutional layers alone is computationally inefficient for modeling long-range dependencies in images.
	\end{itemize}
	\begin{figure}
		\centering
		\includegraphics[width=0.9\linewidth]{figs/conv-vs-sa}
	\end{figure}
	\myfootnotewithlink{https://lilianweng.github.io/lil-log/2018/06/24/attention-attention.html}{image credit: https://lilianweng.github.io/lil-log/2018/06/24/attention-attention.html}
\end{frame}
%=======
\begin{frame}{Self-Attention GAN}
	\begin{figure}
		\centering
		\includegraphics[width=0.9\linewidth]{figs/self-attention}
	\end{figure}
	\vspace{-0.2cm}
	\begin{itemize}
		\item $\bx$ -- feature vector for one feature location.
		\item $N$ -- number of feature locations.
	\end{itemize}
	\vspace{0.5cm}
	\[
		\mathbf{f}(\bx) = \bW_f \bx, \quad \mathbf{g}\bx = \bW_g\bx, \quad \mathbf{h}\bx = \bW_h\bx, \quad \mathbf{v}\bx = \bW_v\bx
	\]
	\[
		s_{ij} = \mathbf{f}(\bx_i)^T \mathbf{g}(\bx_j), \quad a_{ij} = \frac{\exp{s_{ij}}}{\sum_{i=1}^N \exp{s_{ij}}}, \quad \textbf{o}_j = \textbf{v}\left( \sum_{i=1}^N a_{ij} \mathbf{h}(\bx_i) \right)
	\]
	\myfootnotewithlink{https://arxiv.org/abs/1805.08318}{Zhang H. et al. Self-Attention Generative Adversarial Networks, 2018}
\end{frame}
%=======
\begin{frame}{Self-Attention GAN}
	\begin{block}{Technical Details}
		\begin{itemize}
			\item Hinge loss for training.
			\item SpectralNorm in both the generator and the discriminator.
			\item Separate learning rates for the generator and the discriminator.
		\end{itemize}
	\end{block}
	\begin{figure}
		\centering
		\includegraphics[width=0.85\linewidth]{figs/sa_results2}
	\end{figure}
	\vspace{-0.3cm}
	\begin{block}{Visualization of attention maps}
		\begin{figure}
			\centering
			\includegraphics[width=\linewidth]{figs/sa_maps}
		\end{figure}
	\end{block}

	\myfootnotewithlink{https://arxiv.org/abs/1805.08318}{Zhang H. et al. Self-Attention Generative Adversarial Networks, 2018}
\end{frame}
%=======
\begin{frame}{BigGAN}
		\begin{block}{Technical Details}
			\begin{itemize}
				\item Hinge loss.
				\item Self-Attention GAN baseline.
				\item \textbf{Orthogonal regularization}
				\vspace{-0.2cm}
				\[
					\| \bW^T \bW - \bI \|^2 \quad \Rightarrow \quad \| \bW^T \bW - \text{diag}(\bW^T \bW) \|^2
				\]
				\vspace{-0.8cm}
				\item \textbf{Truncation trick.} Components of $\bz \sim \cN(0, \bI)$ which fall outside a predefined range are resampled.
			\end{itemize}
		\end{block}
	\begin{figure}
		\centering
		\includegraphics[width=\linewidth]{figs/biggan_results}
	\end{figure}
	
	\myfootnotewithlink{https://arxiv.org/abs/1809.11096}{Brock A., Donahue J., Simonyan K. Large Scale GAN Training for High Fidelity Natural Image Synthesis, 2018}
\end{frame}
%=======
\begin{frame}{BigGAN}
	\begin{block}{Samples (512x512)}
		\begin{figure}
			\centering
			\includegraphics[width=\linewidth]{figs/biggan_samples}
		\end{figure}
	\end{block}
	\vspace{-0.4cm}
	\begin{block}{Interpolations}
		\begin{figure}
			\centering
			\includegraphics[width=\linewidth]{figs/biggan_interpolations}
		\end{figure}
	\end{block}
	\myfootnotewithlink{https://arxiv.org/abs/1809.11096}{Brock A., Donahue J., Simonyan K. Large Scale GAN Training for High Fidelity Natural Image Synthesis, 2018}
\end{frame}
%=======
\begin{frame}{Progressive Growing GAN}
	\begin{block}{Problems with HR image generation}
		\begin{itemize}
			\item Disjoint manifolds $\Rightarrow$ gradient problem.
			\item Small minibatch $\Rightarrow$ training instability.
		\end{itemize}
	\end{block}
	\begin{block}{Solution}
		Grow both the generator and discriminator progressively, starting from LR images, and add new layers that introduce higher-resolution details as the training progresses. 
		\begin{itemize}
			\item Train GAN which generate 4x4 images (just 2 convolutions for G and D).
			\item Add upsampling layers to G, downsampling layers to D.
			\item Train GAN which generate 8x8 images.
			\item etc.
		\end{itemize}
	\end{block}
	
	\myfootnotewithlink{https://arxiv.org/abs/1710.10196}{Karras T. et al. Progressive Growing of GANs for Improved Quality, Stability, and Variation, 2017}
\end{frame}
\begin{frame}{Progressive Growing GAN}
	\begin{figure}
		\centering
		\includegraphics[width=0.8\linewidth]{figs/pggan_arch}
	\end{figure}
	\begin{figure}
		\centering
		\includegraphics[width=0.6\linewidth]{figs/pggan_fadein}
	\end{figure}

	\myfootnotewithlink{https://arxiv.org/abs/1710.10196}{Karras T. et al. Progressive Growing of GANs for Improved Quality, Stability, and Variation, 2017}
\end{frame}
%=======
\begin{frame}{Progressive Growing GAN}
	\begin{block}{Samples (1024x1024)}
		\begin{figure}
			\centering
			\includegraphics[width=\linewidth]{figs/pggan_samples}
		\end{figure}
	\end{block}

	\myfootnotewithlink{https://arxiv.org/abs/1710.10196}{Karras T. et al. Progressive Growing of GANs for Improved Quality, Stability, and Variation, 2017}
\end{frame}
%=======
\begin{frame}{StyleGAN}
	\begin{itemize}
		\item Generating of HR images is hard.
		\item Progressive growing greatly simplifies the task.
		\item The ability to control specific features of the generated image is very limited.
	\end{itemize}
	\begin{block}{Face image features}
		\begin{itemize}
			\item Coarse (pose, general hair style, face shape). Resolution $4^2 - 8^2$.
			\item Middle (finer facial features, hair style, eyes open/closed). Resolution $16^2 - 32^2$.
			\item Fine (color scheme (eye, hair and skin) and micro features). Resolution $64^2 - 1024^2$.
		\end{itemize}
	\end{block}
	\myfootnotewithlink{https://arxiv.org/abs/1812.04948}{Karras T., Laine S., Aila T. A Style-Based Generator Architecture for Generative Adversarial Networks, 2018}
\end{frame}
%=======
\begin{frame}{StyleGAN}
	\begin{block}{Step 1: Mapping Network}
		\begin{itemize}
			\item Generator input is likely to be \textbf{disentangled}.  Each component of input vector $\bz$ should be responsible for one generative factor.
			\item Mapping network $f: \cZ \rightarrow \cW$ is used to reduce correlations between components of~$\bz$.
		\end{itemize}
		\begin{minipage}[t]{0.6\columnwidth}
			\begin{figure}
				\centering
				\includegraphics[width=0.98\linewidth]{figs/stylegan_mapping}
			\end{figure}
		\end{minipage}%
		\begin{minipage}[t]{0.38\columnwidth}
			\begin{figure}
				\centering
				\includegraphics[width=1.0\linewidth]{figs/stylegan_curved}
			\end{figure}
		\end{minipage}
	\vspace{0.3cm}
	\end{block}

	\myfootnotewithlink{https://arxiv.org/abs/1812.04948}{Karras T., Laine S., Aila T. A Style-Based Generator Architecture for Generative Adversarial Networks, 2018}
\end{frame}
%=======
\begin{frame}{StyleGAN}
	\begin{block}{Step 2: Style modulation}
		\begin{itemize}
			\item Adaptive Instance Normalization transfers the $\bw$ vector to the synthesis Network.
			\item The module is added to each resolution to define the visual expression of the features.
		\end{itemize}
		\begin{figure}
			\centering
			\includegraphics[width=1.0\linewidth]{figs/stylegan_adain}
		\end{figure}
	\end{block}

	\myfootnotewithlink{https://arxiv.org/abs/1812.04948}{Karras T., Laine S., Aila T. A Style-Based Generator Architecture for Generative Adversarial Networks, 2018}
\end{frame}
%=======
\begin{frame}{StyleGAN}
	\begin{block}{Step 3: Remove traditional input}
		Mapping network provides stochasticity to different stages of the synthesis network. Input of the synthesis network is a trainable vector.
		\begin{figure}
			\centering
			\includegraphics[width=0.55\linewidth]{figs/stylegan_input}
		\end{figure}
	\end{block}

	\myfootnotewithlink{https://arxiv.org/abs/1812.04948}{Karras T., Laine S., Aila T. A Style-Based Generator Architecture for Generative Adversarial Networks, 2018}
\end{frame}
%=======
\begin{frame}{StyleGAN}
	\begin{block}{Step 4: Stochastic variation}
		Inject random noise to add small aspects, such as freckles, exact placement of hairs, wrinkles, features which make the image more realistic and increase the variety of outputs.
		\begin{figure}
			\centering
			\includegraphics[width=0.9\linewidth]{figs/stylegan_noise}
		\end{figure}
	\end{block}

	\myfootnotewithlink{https://arxiv.org/abs/1812.04948}{Karras T., Laine S., Aila T. A Style-Based Generator Architecture for Generative Adversarial Networks, 2018}
\end{frame}
%=======
\begin{frame}{StyleGAN}
	\begin{block}{Step 4: Style Mixing}
		\vspace{-0.33cm}
		\begin{figure}
			\centering
			\includegraphics[width=0.8\linewidth]{figs/stylegan_mix_reg}
		\end{figure}
	\begin{itemize}
		\item Makes different levels of synthesis network to be independent.
		\item Allows to couple diffirent styles.
	\end{itemize}
	\end{block}

	\myfootnotewithlink{https://arxiv.org/abs/1812.04948}{Karras T., Laine S., Aila T. A Style-Based Generator Architecture for Generative Adversarial Networks, 2018}
\end{frame}
%=======
\begin{frame}{StyleGAN}
	\begin{figure}
		\centering
		\includegraphics[width=0.8\linewidth]{figs/stylegan_scheme}
	\end{figure}

	\myfootnotewithlink{https://arxiv.org/abs/1812.04948}{Karras T., Laine S., Aila T. A Style-Based Generator Architecture for Generative Adversarial Networks, 2018}
\end{frame}
%=======
\begin{frame}{StyleGAN}
	\begin{block}{Truncation trick}
		\vspace{-0.2cm}
		\[
			\bw' = \hat{\bw} - \psi \cdot (\bw - \hat{\bw}), \quad \hat{\bw} = \bbE_{\bz} p(f(\bz))
		\]
		\vspace{-0.2cm}
		\begin{itemize}
			\item Constant $\psi$ is a tradeoff between diversity and fidelity. 
			\item $\psi=0.7$ is used for most of the results.
			\item Truncation is done only at the low-resolution layers.
		\end{itemize}
		\begin{figure}
			\centering
			\includegraphics[width=0.8\linewidth]{figs/stylegan_truncation}
		\end{figure}
	\end{block}

	\myfootnotewithlink{https://arxiv.org/abs/1812.04948}{Karras T., Laine S., Aila T. A Style-Based Generator Architecture for Generative Adversarial Networks, 2018}
\end{frame}
%=======
\begin{frame}{StyleGAN}
	\begin{block}{Results}
		\vspace{-0.2cm}
		\begin{figure}
			\centering
			\includegraphics[width=0.6\linewidth]{figs/stylegan_results}
		\end{figure}
	\vspace{-0.3cm}
	\end{block}
	\begin{block}{Samples (1024x1024)}
		\begin{figure}
			\centering
			\includegraphics[width=0.8\linewidth]{figs/stylegan_samples}
		\end{figure}
	\vspace{-0.1cm}
	\end{block}

	\myfootnotewithlink{https://arxiv.org/abs/1812.04948}{Karras T., Laine S., Aila T. A Style-Based Generator Architecture for Generative Adversarial Networks, 2018}
\end{frame}
%=======
\begin{frame}{StyleGAN}
		\begin{figure}
			\centering
			\includegraphics[width=0.6\linewidth]{figs/stylegan_mix}
		\end{figure}

	\myfootnotewithlink{https://arxiv.org/abs/1812.04948}{Karras T., Laine S., Aila T. A Style-Based Generator Architecture for Generative Adversarial Networks, 2018}
\end{frame}
%=======
\begin{frame}{Summary}
\end{frame}
%=======
\end{document} 