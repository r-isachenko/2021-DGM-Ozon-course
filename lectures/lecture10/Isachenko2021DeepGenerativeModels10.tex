\documentclass{beamer}
\usepackage[utf8]{inputenc}
\usepackage{graphicx, epsfig}
\usepackage{amsmath,mathrsfs,amsfonts,amssymb}
%\usepackage{subfig}
\usepackage{floatflt}
\usepackage{epic,ecltree}
\usepackage{mathtext}
\usepackage{fancybox}
\usepackage{fancyhdr}
\usepackage{multirow}
\usepackage{enumerate}
\usepackage{epstopdf}
\usepackage{multicol}
\usepackage{algorithm}
\usepackage[noend]{algorithmic}
\usepackage{tikz}
\usepackage{blindtext}
\usetheme{default}%{Singapore}%{Warsaw}%{Warsaw}%{Darmstadt}
\usecolortheme{default}
\setbeamerfont{title}{size=\Huge}
\setbeamertemplate{footline}[page number]{}
\setbeamerfont{title}{size=\Huge}
\beamertemplatenavigationsymbolsempty

\input{../utils/newcommands}

\newcommand{\createdgmtitle}[1]{\title[\hbox to 56mm{Deep Generative Models  \hfill\insertframenumber\,/\,\inserttotalframenumber}]
	{\vspace{1cm} \\ Deep Generative Models \\ Lecture #1 \\ \vspace{-0.5cm}}
	\author{Roman Isachenko \\ \vspace{-0.5cm}}
	\institute{\includegraphics[width=3cm]{../utils/ozonmasterslogo}
	\\Ozon Masters
	}
	\date{Spring, 2021}
}

\newcommand\myfootnote[1]{%
  \tikz[remember picture,overlay]
  \draw (current page.south west) +(1in + \oddsidemargin,0.5em)
  node[anchor=south west,inner sep=0pt]{\parbox{\textwidth}{%
      \rlap{\rule{10em}{0.4pt}}\raggedright\scriptsize#1}};}

\newcommand\myfootnotewithlink[2]{%
  \tikz[remember picture,overlay]
  \draw (current page.south west) +(1in + \oddsidemargin,0.5em)
  node[anchor=south west,inner sep=0pt]{\parbox{\textwidth}{%
      \rlap{\rule{10em}{0.4pt}}\raggedright\scriptsize\href{#1}{\textit{#2}}}};}
\createdgmtitle{10}
%--------------------------------------------------------------------------------
\begin{document}
%--------------------------------------------------------------------------------
\begin{frame}
%\thispagestyle{empty}
\titlepage
\end{frame}
%=======
\begin{frame}{Informal theoretical results}
	\begin{itemize}
		\footnotesize
		\item Since $\bz$ usually has lower dimensionality compared to $\bx$, manifold $G(\bz)$ has a measure 0 in $\bx$ space. Hence, support of $p(\bx | \btheta)$ lies on low-dimensional manifold.
		\item Distribution of real images $\pi(\bx)$ is also concentrated on a low dimensional manifold.
		\begin{figure}
			\centering
			\includegraphics[width=0.5\linewidth]{figs/low_dim_manifold}
		\end{figure}
		\item If $\pi(\bx)$ and $p(\bx | \btheta)$ have disjoint supports, then there is a smooth optimal discrimator. We are not able to learn anything by backproping through it.
		\item For such low-dimensional disjoint manifolds
		\vspace{-0.1cm}
		\[
			KL(\pi || p) = KL(p || \pi) = \infty, \quad JSD(\pi || p) = \log 2
		\]
		\vspace{-0.7cm}
		\item Adding continuous noise to the inputs of the discriminator smoothes the distributions of the probability mass.
	\end{itemize}
	\myfootnote{\href{https://arxiv.org/abs/1904.08994}{Weng L. From GAN to WGAN, 2019} \\ 
	\href{https://arxiv.org/abs/1701.04862}{Arjovsky M., Bottou L. Towards Principled Methods for Training Generative Adversarial Networks, 2017}}
\end{frame}
%=======
\begin{frame}{Wasserstein distance (discrete)}
	Also called Earth Mover's distance.
	The minimum cost of moving and transforming a pile of dirt in the shape of one probability distribution to the shape of the other distribution.
	\begin{figure}
		\centering
		\includegraphics[width=.9\linewidth]{figs/EM_distance_discrete}
	\end{figure}
	\[
		W(P, Q) = 2 \text{(step 1)} + 2 \text{(step 2)} + 1 \text{(step 3)}  = 5
	\]
	\myfootnotewithlink{https://arxiv.org/abs/1904.08994}{Weng L. From GAN to WGAN, 2019} 
	
\end{frame}
%=======
\begin{frame}{Wasserstein distance}
	\[
		W(\pi, p) = \inf_{\gamma \in \prod(\pi, p)} \bbE_{(\bx, \by) \sim \gamma} \| \bx - \by \| =  \inf_{\gamma \in \prod(\pi, p)} \int \| \bx - \by \| \gamma (\bx, \by) d \bx d \by
	\]
	\begin{itemize}
		\item $\prod(\pi, p)$ -- the set of all joint distributions $\gamma (\bx, \by)$ with marginals $\pi$ and $p$ ($\int \gamma(\bx, \by) d \bx = p(\by)$, $\int \gamma(\bx, \by) d \by = \pi(\bx)$)
		\item $\gamma(\bx, \by)$ -- transportation plan (the amount of "dirt" that should be transported from point $\bx$ to point $\by$).
		\item $\gamma(\bx, \by)$ -- the amount, $\|\bx - \by \|$-- the distance.
	\end{itemize}
	For better understanding of transportation plan function $\gamma$, try to write down the plan for previous discrete case.
	\myfootnotewithlink{https://arxiv.org/abs/1701.07875}{Arjovsky M., Chintala S., Bottou L. Wasserstein GAN, 2017}
\end{frame}
%=======
\begin{frame}{Wasserstein distance vs KL vs JSD}
	
	\begin{minipage}[t]{0.48\columnwidth}
		\vspace{0.1cm}
		Consider 2d distributions
		\[
			\pi(x, y) = (0, U[0,1])
		\]	
		\[
			p(x, y | \theta) = (\theta, U[0, 1])
		\]
	\end{minipage}%
	\begin{minipage}[t]{0.52\columnwidth}
		\begin{figure}
			\centering
			\includegraphics[width=0.8\linewidth]{figs/w_kl_jsd}
		\end{figure}
	\end{minipage}
	\begin{itemize}
		\footnotesize
		\item $\theta = 0$.
		Distributions are the same 
		\[
			KL(\pi || p) = KL(p || \pi) = JSD(p || \pi) = W(\pi, p) = 0
		\]
		\item $\theta \neq 0$
		\[
			KL(\pi || p) = \int_{U[0, 1]} 1 \log \frac{1}{0} d y = \infty = KL(p || \pi)
		\]
		\[
			JSD(\pi || p) = \frac{1}{2}\left( \int_{U[0, 1]}1 \log \frac{1}{1/2} dy + \int_{U[0, 1]}1 \log \frac{1}{1/2} dy \right) = \log 2
		\]
		\[
			W(\pi, p) = |\theta|
		\]
	\end{itemize}
	
	\myfootnote{\href{https://arxiv.org/abs/1904.08994}{Weng L. From GAN to WGAN, 2019} \\ 
	\href{https://arxiv.org/abs/1701.07875}{Arjovsky M., Chintala S., Bottou L. Wasserstein GAN, 2017}}
\end{frame}
%=======
\begin{frame}{Wasserstein distance vs KL vs JSD}
	\begin{block}{Theorem 1}
		Let $G(\bz, \btheta)$ be any feedforward neural network, and $p(\bz)$ a prior over $\bz$ such that $\bbE_{\bz \sim p(\bz)}
		\|\bz\| < \infty$. Then therefore $W(\pi, p)$ is continuous everywhere and differentiable almost everywhere.
	\end{block}
	\begin{block}{Theorem 2}
		Let $\pi$ be a distribution on a compact space $\cX$ and $\{p_t\}_{t=1}^\infty$ be a sequence of distributions on $\cX$. 
		\begin{align}
			KL(\pi || p_t) &\rightarrow 0 \, (\text{or }KL (p_t || \pi) \rightarrow 0) \\
			JSD(\pi || p_t) &\rightarrow 0 \\
			W(\pi || p_t) &\rightarrow 0
		\end{align}
		
		Then, considering limits as $t \rightarrow \infty$, (1) implies (2), (2) implies (3).
	\end{block}
	\myfootnotewithlink{https://arxiv.org/abs/1701.07875}{Arjovsky M., Chintala S., Bottou L. Wasserstein GAN, 2017}
\end{frame}
%=======
\begin{frame}{Wasserstein distance}
	\[
	W(\pi || p) = \inf_{\gamma \in \prod(\pi, p)} \bbE_{(\bx, \by) \sim \gamma} \| \bx - \by \| =  \inf_{\gamma \in \prod(\pi, p)} \int \| \bx - \by \| \gamma (\bx, \by) d \bx d \by
	\]
	\begin{itemize}
		\item $\prod(\pi, p)$ -- the set of all joint distributions $\gamma (\bx, \by)$ with marginals $\pi$ and $p$ ($\int \gamma(\bx, \by) d \bx = p(\by)$, $\int \gamma(\bx, \by) d \by = \pi(\bx)$)
		\item $\gamma(\bx, \by)$ -- transportation plan (the amount of "dirt" that should be transported from point $\bx$ to point $\by$).
		\item $\gamma(\bx, \by)$ -- the amount, $\|\bx - \by \|$-- the distance.
	\end{itemize}
	For better understanding of transportation plan function $\gamma$, try to write down the plan for previous discrete case.

	\myfootnotewithlink{https://arxiv.org/abs/1701.07875}{Arjovsky M., Chintala S., Bottou L. Wasserstein GAN, 2017}
\end{frame}
%=======
\begin{frame}{Wasserstein distance vs KL vs JSD}
	\begin{block}{Theorem 1}
		Let $G(\bz, \btheta)$ be any feedforward neural network, and $p(\bz)$ a prior over $\bz$ such that $\bbE_{\bz \sim p(\bz)}
		\|\bz\| < \infty$. Then therefore $W(\pi || p)$ is continuous everywhere and differentiable almost everywhere.
	\end{block}
	\begin{block}{Theorem 2}
		Let $\pi$ be a distribution on a compact space $\cX$ and $\{p_t\}_{t=1}^\infty$ be a sequence of distributions on $\cX$. 
		\begin{align}
			KL(\pi || p_t) &\rightarrow 0 \, (\text{or }KL (p_t || \pi) \rightarrow 0) \\
			JSD(\pi || p_t) &\rightarrow 0 \\
			W(\pi || p_t) &\rightarrow 0
		\end{align}
		
		Then, considering limits as $t \rightarrow \infty$, (1) implies (2), (2) implies (3).
	\end{block}

	\myfootnotewithlink{https://arxiv.org/abs/1701.07875}{Arjovsky M., Chintala S., Bottou L. Wasserstein GAN, 2017}
\end{frame}
%=======
\begin{frame}{Wasserstein GAN}
	\begin{block}{Wasserstein distance}
		\vspace{-0.4cm}
		\[
			W(\pi || p) = \inf_{\gamma \in \prod(\pi, p)} \bbE_{(\bx, \by) \sim \gamma} \| \bx - \by \| =  \inf_{\gamma \in \prod(\pi, p)} \int \| \bx - \by \| \gamma (\bx, \by) d \bx d \by
		\]
	\end{block}
	The infimum across all possible joint distributions in $\prod(\pi, p)$ is intractable.
	\begin{block}{Kantorovich-Rubinstein duality}
		\[
			W(\pi || p) = \frac{1}{K} \max_{\| f \|_L \leq K} \left[ \bbE_{\bx \sim \pi} f(\bx)  - \bbE_{\bx \sim p} f(\bx)\right],
		\]
		where $\| f \|_L \leq K$ are $K-$Lipschitz continuous functions ($f: \cX \rightarrow \bbR$)
		\[
			|f(\bx_1) - f(\bx_2)| \leq K \| \bx_1 - \bx_2 \|, \quad \text{for all } \bx_1, \bx_2 \in \cX.
		\]
	\end{block}

	\myfootnotewithlink{https://arxiv.org/abs/1701.07875}{Arjovsky M., Chintala S., Bottou L. Wasserstein GAN, 2017}
\end{frame}
%=======
\begin{frame}{Wasserstein GAN}
		\begin{block}{Kantorovich-Rubinstein duality}
		\[
			W(\pi || p) = \frac{1}{K} \max_{\| f \|_L \leq K} \left[ \bbE_{\bx \sim \pi} f(\bx)  - \bbE_{\bx \sim p} f(\bx)\right],
		\]
	\end{block}
	\begin{itemize}
		\item Now we have to ensure that $f$ is $K$-Lipschitz continuous.
		\item Let $f(\bx, \phi)$ be parametrized by parameters $\bphi$.
		\item If parameters $\phi$ lie in a compact set $\boldsymbol{\Phi}$ then $f(\bx, \bphi)$ will be $K$-Lipschitz continuous function. 
		\item Let the parameters be clamped to a fixed box $\boldsymbol{\Phi} \in [-0.01, 0.01]^d$ after each gradient update.
	\end{itemize}
	{\footnotesize
	\[
		 \max_{\bphi \in \boldsymbol{\Phi}} \left[ \bbE_{\bx \sim \pi} f(\bx, \bphi)  - \bbE_{\bx \sim p} f(\bx, \bphi )\right] \leq  \max_{\| f \|_L \leq K} \left[ \bbE_{\bx \sim \pi} f(\bx)  - \bbE_{\bx \sim p} f(\bx)\right] = K \cdot W(\pi || p)
	\]}

	\myfootnotewithlink{https://arxiv.org/abs/1701.07875}{Arjovsky M., Chintala S., Bottou L. Wasserstein GAN, 2017}
\end{frame}
%=======
\begin{frame}{Wassestein GAN}
	\begin{block}{Vanilla GAN objective}
		\vspace{-0.3cm}
		\[
			\min_{G} \max_D \bbE_{\pi(\bx)} \log D(\bx) + \bbE_{p(\bz)} \log (1 - D(G(\bz)))
		\]
	\end{block}
	\begin{block}{WGAN objective}
		\vspace{-0.6cm}
		\[
		\min_{G} W(\pi || p) = \min_{G} \max_{\bphi \in \boldsymbol{\Phi}} \left[ \bbE_{\pi(\bx)} f(\bx, \bphi)  - \bbE_{p(\bz)} f(G(\bz), \bphi )\right].
		\]
	\end{block}
	\begin{itemize}
		\item Discriminator $D$ is similar to the function $f$, but not the same (it is not a classifier anymore). In the WGAN model, function $f$ is usually called $\textit{critic}$.
		\item "Weight clipping is a clearly terrible way to enforce a Lipschitz constraint". If the clipping parameter is large, it is hard to train the critic till optimality. If the clipping parameter is too small, it could lead to vanishing gradients.
	\end{itemize}

	\myfootnotewithlink{https://arxiv.org/abs/1701.07875}{Arjovsky M., Chintala S., Bottou L. Wasserstein GAN, 2017}
\end{frame}
%=======
\begin{frame}{Wasserstein GAN}
	\begin{figure}
		\centering
		\includegraphics[width=1.0\linewidth]{figs/wgan_pseudocode}
	\end{figure}

	\myfootnotewithlink{https://arxiv.org/abs/1701.07875}{Arjovsky M., Chintala S., Bottou L. Wasserstein GAN, 2017}
\end{frame}
%=======
\begin{frame}{Wasserstein GAN}
	\begin{figure}
		\centering
		\includegraphics[width=0.8\linewidth]{figs/wgan_toy}
	\end{figure}

	\myfootnotewithlink{https://arxiv.org/abs/1701.07875}{Arjovsky M., Chintala S., Bottou L. Wasserstein GAN, 2017}
\end{frame}
%=======
\begin{frame}{Wasserstein GAN}
	\begin{minipage}[t]{0.49\columnwidth}
		\begin{figure}
			\centering
			\includegraphics[width=1.0\linewidth]{figs/dcgan_quality}
		\end{figure}
	\end{minipage}%
	\begin{minipage}[t]{0.49\columnwidth}
		\begin{figure}
			\centering
			\includegraphics[width=1.0\linewidth]{figs/wgan_quality}
		\end{figure}
	\end{minipage}
	\begin{itemize}
		\item $JSD$ correlates poorly with the sample quality. Stays contast nearly maximum value $\log 2 \approx 0.69$.
		\item $W$ is highly correlated with the sample quality. 
	\end{itemize}

	\myfootnotewithlink{https://arxiv.org/abs/1701.07875}{Arjovsky M., Chintala S., Bottou L. Wasserstein GAN, 2017}
\end{frame}
%=======
\begin{frame}{Wasserstein GAN}
	\begin{block}{WGAN converged without batch norm and constant number of filters}
		\begin{figure}
			\centering
			\includegraphics[width=1.0\linewidth]{figs/wgan_convergence}
		\end{figure}
	\end{block}
	\begin{block}{"In no experiment did we see evidence of mode collapse for the WGAN algorithm."}
		\begin{figure}
			\centering
			\includegraphics[width=1.0\linewidth]{figs/wgan_mode_collapse}
		\end{figure}
	\end{block}

	\myfootnotewithlink{https://arxiv.org/abs/1701.07875}{Arjovsky M., Chintala S., Bottou L. Wasserstein GAN, 2017}
\end{frame}
%=======
\begin{frame}{Wasserstein GAN with Gradient Penalty}
	
	\begin{minipage}[t]{0.45\columnwidth}
		\vspace{0.2cm}
		The generator distribution is fixed and equal to the real distribution + Gaussian noise. \\
		Problems with weight clipping:
		\begin{itemize}
			\item The critic ignores higher moments of the data distribution.
			\item The gradients either grow or decay exponentially.
		\end{itemize}
	Gradient penalty makes the gradients  more stable.
	\end{minipage}%
	\begin{minipage}[t]{0.52\columnwidth}
		\begin{figure}
			\centering
			\includegraphics[width=0.95\linewidth]{figs/wgan_gp_toy}
		\end{figure}
		\begin{figure}
			\centering
			\includegraphics[width=0.95\linewidth]{figs/wgan_gp_weights}
		\end{figure}
	\end{minipage}
	
	\myfootnotewithlink{https://arxiv.org/abs/1704.00028}{Gulrajani I. et al. Improved Training of Wasserstein GANs, 2017}
\end{frame}
%=======
\begin{frame}{Wasserstein GAN with Gradient Penalty}
	\begin{block}{Theorem}
		Let $\pi(\bx)$ and $p(\bx)$ be two distribution in $\cX$, a compact metric space. Then, there is 1-Lipschitz function $f^*$ which is the optimal solution of 
		\[
			\max_{\| f \|_L \leq 1} \left[ \bbE_{\bx \sim \pi} f(\bx)  - \bbE_{\bx \sim p} f(\bx)\right].
		\]
		Let $\gamma$ be the optimal transportation plan between $\pi(\bx)$ and $p(\bx)$. Then, if $f^*$ is differentiable $\gamma(x = y) = 0$ and $\bx_t = t \bx + (1 - t) \by$ with $t \in [0, 1]$ it holds that
		\[
			\bbP_{(\bx, \by) \sim \gamma} \left[ \nabla f^*(\bx_t) = \frac{\by - \bx_t}{\| \by - \bx_t \|} \right] = 1.
		\]
	\end{block}
	\begin{block}{Corollary}
		$f^*$ has gradient norm 1 almost everywhere under $\pi(\bx)$ and $p(\bx)$.
	\end{block}

	\myfootnotewithlink{https://arxiv.org/abs/1704.00028}{Gulrajani I. et al. Improved Training of Wasserstein GANs, 2017}
\end{frame}
%=======
\begin{frame}{Wasserstein GAN with Gradient Penalty}
	A differentiable function is 1-Lipschtiz if and only if it has gradients with norm at most 1 everywhere.
	\begin{block}{Gradient penalty}
		\[
			W(\pi || p) = \underbrace{\bbE_{\bx \sim \pi} f(\bx)  - \bbE_{\bx \sim p} f(\bx)}_{\text{original critic loss}} + \lambda \underbrace{\bbE_{\hat{\bx}} \left[ \left( \| \nabla_{\hat{\bx}} f(\hat{\bx}) \|_2 - 1 \right) ^ 2\right]}_{\text{gradient penalty}},
		\]
	\end{block}
	\begin{itemize}
		\item Samples $\hat{\bx}$ are uniformly sampled along straight lines between pairs of points from the data distribution $\pi(\bx)$ and the generator distribution $p(\bx | \btheta)$.
		\item Enforcing the unit gradient norm constraint everywhere is intractable, it turns out to be sifficient to enforce it only along these straight lines.
	\end{itemize}

	\myfootnotewithlink{https://arxiv.org/abs/1704.00028}{Gulrajani I. et al. Improved Training of Wasserstein GANs, 2017}
\end{frame}
%=======
\begin{frame}{Wasserstein GAN with Gradient Penalty}
	
	\begin{figure}
		\centering
		\includegraphics[width=1.0\linewidth]{figs/wgan_gp_pseudocode}
	\end{figure}

	\myfootnotewithlink{https://arxiv.org/abs/1704.00028}{Gulrajani I. et al. Improved Training of Wasserstein GANs, 2017}
\end{frame}
%=======
\begin{frame}{Wasserstein GAN with Gradient Penalty}
	\begin{figure}
		\centering
		\includegraphics[width=0.9\linewidth]{figs/wgan_gp_convergence}
	\end{figure}
	\begin{figure}
		\centering
		\includegraphics[width=0.65\linewidth]{figs/wgan_gp_model_space}
	\end{figure}
	\begin{figure}
		\centering
		\includegraphics[width=0.65\linewidth]{figs/wgan_gp_wgan}
	\end{figure}

	\myfootnotewithlink{https://arxiv.org/abs/1704.00028}{Gulrajani I. et al. Improved Training of Wasserstein GANs, 2017}
\end{frame}
%=======
\begin{frame}{Wasserstein GAN with Gradient Penalty}
	\begin{figure}
		\centering
		\includegraphics[width=0.95\linewidth]{figs/wgan_gp_results}
	\end{figure}

	\myfootnotewithlink{https://arxiv.org/abs/1704.00028}{Gulrajani I. et al. Improved Training of Wasserstein GANs, 2017}
\end{frame}
%=======
\begin{frame}{Spectral Normalization GAN}
	How else could we enforce Lipschitzness?
	\begin{block}{Fact 1}
		\vspace{-0.3cm}
		\[
			\| \mathbf{g} \|_L = \sup_\bx \sigma( \nabla \mathbf{g}(\bx))
		\]
	\end{block}
	Here $\sigma(\bA)$~-- spectral norm of matrix $\bA$.
	\[
		\sigma(\bA) = \max_{\bh \neq 0} \frac{\|\bA \bh\|_2}{\|\bh\|_2} = \max_{\|\bh\|_2 \leq 1} \| \bA \bh \|_2 = \lambda_{\text{max}}(\bA),
	\]
	where $\lambda_{\text{max}}(\bA)$ is the largest singular value of $\bA$.
	\begin{block}{Fact 2}
		\vspace{-0.3cm}
		\[
			\| \mathbf{g}_1 \circ \mathbf{g}_2 \|_L \leq \| \mathbf{g}_1 \|_L \cdot \| \mathbf{g}_2\|_L
		\]
	\end{block}
	\myfootnotewithlink{https://arxiv.org/abs/1802.05957}{Miyato T. et al. Spectral Normalization for Generative Adversarial Networks, 2018}
\end{frame}
%=======
\begin{frame}{Spectral Normalization GAN}
	Let consider the critic $f(\bx, \bphi)$ of the following form:
	\[
		f(\bx, \bphi) = \bW_{K+1} a_K (\bW_K a_{K-1}(\dots a_1(\bW_1 \bx) \dots)).
	\]
	This feedforward network is a superposition of simple functions.
	\begin{itemize}
		\item $a_k$ is a pointwise nonlinearities. We assume that $\| a_k \|_L = 1$ (it holds for ReLU).
		\item $\mathbf{g}(\bh) = \bW \bh$ is a linear transformation.
		\[
			\| \mathbf{g} \|_L = \sup_\bx \sigma( \nabla \mathbf{g}(\bx)) = \sigma(\bW).
		\]
	\end{itemize}
	\vspace{-0.5cm}
	\begin{block}{Critic spectral norm}
		\vspace{-0.5cm}
		\[
			\| f \|_L \leq \prod_{k=1}^{K+1} \sigma(\bW_k).
		\]
	\end{block}
	If we replace the weights in the critic by $\bW^{SN}_k = \bW_k / \sigma(\bW_k)$, we will get $\| f\|_L \leq 1.$ \\
	
	\myfootnotewithlink{https://arxiv.org/abs/1802.05957}{Miyato T. et al. Spectral Normalization for Generative Adversarial Networks, 2018}
\end{frame}
%=======
\begin{frame}{Spectral Normalization GAN}
	 If we apply singular value decomposition to compute the $\sigma(\bW)$ at each round of the algorithm, the algorithm becomes computationally heavy.
	 \begin{block}{Power iteration}
	 	\begin{itemize}
	 		\item $\bu$ -- random vector.
	 		\item repeat
	 		\[
	 			\bv = \frac{\bW^T \bu}{\| \bW^T \bu \|}, \quad \bu = \frac{\bW \bv}{\| \bW \bv \|}
	 		\]
	 		\item approximate the spectral norm
	 		\[
	 			\sigma(\bW) \approx \bu^T \bW \bv
	 		\]
	 	\end{itemize}
	 \end{block}

	\myfootnotewithlink{https://arxiv.org/abs/1802.05957}{Miyato T. et al. Spectral Normalization for Generative Adversarial Networks, 2018}
\end{frame}
%=======
\begin{frame}{Spectral Normalization GAN}
	\begin{figure}
		\centering
		\includegraphics[width=0.85\linewidth]{figs/sngan_pseudocode}
	\end{figure}
	\begin{figure}
		\centering
		\includegraphics[width=0.85\linewidth]{figs/sngan_fids}
	\end{figure}

	\myfootnotewithlink{https://arxiv.org/abs/1802.05957}{Miyato T. et al. Spectral Normalization for Generative Adversarial Networks, 2018}
\end{frame}
%=======
\begin{frame}{Divergences}
	\begin{block}{What do we have?}
		\begin{itemize}
			\item Forward KL divergence in maximum likelihood estimation
			\item Reverse KL in variational inference
			\item JS divergence in  vanilla gan
			\item Wasserstein distance in WGAN
		\end{itemize}
	\end{block}
	\begin{block}{Divergence minimization}
		\vspace{-0.3cm}
		\[
			\min_p D(\pi || p)
		\]
		\vspace{-0.5cm}
	\end{block}
	\begin{block}{What is a divergence?}
		Let $\cS$ be the set of all possible probability distributions. Then $D: \cS \times \cS \rightarrow \bbR$ is a divergence if 
		\begin{itemize}
			\item $D(\pi || p) \geq 0$ for all $\pi, p \in \cS$;
			\item $D(\pi || p) = 0$ if and only if $\pi \equiv p$.
		\end{itemize}
	\end{block}
\end{frame}
%=======
\begin{frame}{f-divergence family}
	
	\begin{block}{f-divergence}
		\vspace{-0.3cm}
		\[
		D_f(\pi || p) = \bbE_p  f\left( \frac{\pi(\bx)}{p(\bx)} \right)  = \int p(\bx) f\left( \frac{\pi(\bx)}{p(\bx)} \right) d \bx.
		\]
		Here $f: \bbR_+ \rightarrow \bbR$ is a convex, lower-semicontinuous function satisfying $f(1) = 0$.
	\end{block}
	\begin{block}{Fenchel conjugate}
		\vspace{-0.5cm}
		\[
		f^*(t) = \sup_{u \in \text{dom}_f} \left( ut - f(u) \right), \quad f^{**} = f, \quad f(u) = \sup_{t \in \text{dom}_{f^*}} \left( ut - f^*(t) \right)
		\]
		\vspace{-0.5cm}
	\end{block}
	\begin{figure}
		\centering
		\includegraphics[width=0.85\linewidth]{figs/f_divs}
	\end{figure}
	\myfootnotewithlink{https://arxiv.org/abs/1606.00709}{Nowozin S., Cseke B., Tomioka R. f-GAN: Training Generative Neural Samplers using Variational Divergence Minimization, 2016}
\end{frame}
%=======
\begin{frame}{f-divergence family}
	\begin{block}{Variational divergence estimation}
		\vspace{-0.6cm}
		\begin{align*}
			D_f(\pi || p) &= \int p(\bx) f\left( \frac{\pi(\bx)}{p(\bx)} \right) d \bx \\
			& = \int p(\bx) \sup_{t \in \text{dom}_{f^*}} \left( \frac{\pi(\bx)}{p(\bx)} t - f^*(t) \right) d \bx \\
			& = \int \sup_{t \in \text{dom}_{f^*}} \left( \pi(\bx)t - p(\bx) f^*(t) \right) d \bx \\
			& \geq \sup_{T \in \cT} \int \left( \pi(\bx)T(\bx) - p(\bx) f^*(T(\bx)) \right) d \bx \\
			& = \sup_{T \in \cT} \left[\bbE_{\pi}T(\bx) -  \bbE_p f^*(T(\bx)) \right]
		\end{align*}
	\vspace{-0.6cm}
	\end{block}
	Here $\cT: \cX \rightarrow \bbR$ is an arbitrary class of functions.
	
	The lower bound is tight for $T^*(\bx) = f'\left( \frac{\pi(\bx)}{p(\bx)} \right)$.
	\myfootnotewithlink{https://arxiv.org/abs/1606.00709}{Nowozin S., Cseke B., Tomioka R. f-GAN: Training Generative Neural Samplers using Variational Divergence Minimization, 2016}
\end{frame}
%=======
\begin{frame}{f-divergence family}
	\begin{block}{Variational divergence estimation}
		\[
			D_f(\pi || p) \geq \sup_{T \in \cT} \left[\bbE_{\pi}T(\bx) -  \bbE_p f^*(T(\bx)) \right]
		\]
	\end{block}
	\begin{figure}
		\centering
		\includegraphics[width=1.0\linewidth]{figs/f_div_results}
	\end{figure}

	\myfootnotewithlink{https://arxiv.org/abs/1606.00709}{Nowozin S., Cseke B., Tomioka R. f-GAN: Training Generative Neural Samplers using Variational Divergence Minimization, 2016}
\end{frame}
%=======
\begin{frame}{Summary}
	\begin{itemize} 
		\item Wasserstein distance is more appropriate objective function for distribution matching problem.
	\end{itemize}
\end{frame}
%=======
\end{document} 