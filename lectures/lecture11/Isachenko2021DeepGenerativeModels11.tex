\documentclass{beamer}
\usepackage[utf8]{inputenc}
\usepackage{graphicx, epsfig}
\usepackage{amsmath,mathrsfs,amsfonts,amssymb}
%\usepackage{subfig}
\usepackage{floatflt}
\usepackage{epic,ecltree}
\usepackage{mathtext}
\usepackage{fancybox}
\usepackage{fancyhdr}
\usepackage{multirow}
\usepackage{enumerate}
\usepackage{epstopdf}
\usepackage{multicol}
\usepackage{algorithm}
\usepackage[noend]{algorithmic}
\usepackage{tikz}
\usepackage{blindtext}
\usetheme{default}%{Singapore}%{Warsaw}%{Warsaw}%{Darmstadt}
\usecolortheme{default}
\setbeamerfont{title}{size=\Huge}
\setbeamertemplate{footline}[page number]{}
\setbeamerfont{title}{size=\Huge}
\beamertemplatenavigationsymbolsempty

\input{../utils/newcommands}

\newcommand{\createdgmtitle}[1]{\title[\hbox to 56mm{Deep Generative Models  \hfill\insertframenumber\,/\,\inserttotalframenumber}]
	{\vspace{1cm} \\ Deep Generative Models \\ Lecture #1 \\ \vspace{-0.5cm}}
	\author{Roman Isachenko \\ \vspace{-0.5cm}}
	\institute{\includegraphics[width=3cm]{../utils/ozonmasterslogo}
	\\Ozon Masters
	}
	\date{Spring, 2021}
}

\newcommand\myfootnote[1]{%
  \tikz[remember picture,overlay]
  \draw (current page.south west) +(1in + \oddsidemargin,0.5em)
  node[anchor=south west,inner sep=0pt]{\parbox{\textwidth}{%
      \rlap{\rule{10em}{0.4pt}}\raggedright\scriptsize#1}};}

\newcommand\myfootnotewithlink[2]{%
  \tikz[remember picture,overlay]
  \draw (current page.south west) +(1in + \oddsidemargin,0.5em)
  node[anchor=south west,inner sep=0pt]{\parbox{\textwidth}{%
      \rlap{\rule{10em}{0.4pt}}\raggedright\scriptsize\href{#1}{\textit{#2}}}};}
\createdgmtitle{11}
%--------------------------------------------------------------------------------
\begin{document}
%--------------------------------------------------------------------------------
\begin{frame}
%\thispagestyle{empty}
\titlepage
\end{frame}
%=======
\begin{frame}{Disentangled representations}
	\begin{block}{Unsupervised representation learning}
	    Learning an interpretable factorised representation of the independent data generative factors of the world without supervision. 
	\end{block}
	\begin{block}{Disentanglement informal definition}
	A disentangled representation can be defined as one where single latent units are sensitive to changes in single generative factors, while being relatively invariant to changes in other factors. 
	\end{block}
	\begin{block}{Example}
	Model trained on a dataset of 3D objects might learn independent latent units sensitive to single independent data generative factors, such as object identity, position, scale, lighting or colour. 
	\end{block}
	\myfootnotewithlink{https://openreview.net/references/pdf?id=Sy2fzU9gl}{Higgins I. et al. beta-VAE: Learning Basic Visual Concepts with a Constrained Variational Framework, 2017}
\end{frame}
%=======
\begin{frame}{$\beta$-VAE, 2017}
	\begin{block}{Generative process}
	\begin{itemize}
	    \item $p(\bx | \bv, \bw) = \text{Sim}(\bv, \bw)$~-- true world simulator;
	    \item $\bv$~-- conditionally independent factors: $p(\bv | \bx) = \prod_{j=1}^d p(v_j | \bx)$;
	    \item $\bw$~-- conditionally dependent factors. 
	\end{itemize}
	\end{block}
	\begin{block}{Goal}
	Construct an unsupervised deep generative model
	\[
	    p(\bx | \bz) \approx p(\bx | \bv, \bw).
	\]
	\vspace{-0.5cm}
	\begin{itemize}
	    \item Ensure that the inferred latent factors $q(\bz|\bx)$ capture the factors $\bv$ in a disentangled manner. 
	    \item The conditionally dependent factors $\bw$ can remain entangled in a separate subset of $\bz$ that is not used for representing $\bv$. 
	\end{itemize}
	\end{block}
	\myfootnotewithlink{https://openreview.net/references/pdf?id=Sy2fzU9gl}{Higgins I. et al. beta-VAE: Learning Basic Visual Concepts with a Constrained Variational Framework, 2017}
\end{frame}
%=======
\begin{frame}{InfoGAN}
	\begin{block}{GAN objective}
		\vspace{-0.6cm}
		\[
		\min_{G} \max_D V(G, D)
		\]
		\[
		V(G, D)  =  \bbE_{\pi(\bx)} \log D(\bx) + \bbE_{p(\bz)} \log (1 - D(G(\bz)))
		\]
	\end{block}
	Latent vector $\bz$ is not imposed to be disentangled.
	
	InfoGAN decomposes input vector:
	\begin{itemize}
		\item $\bz$ -- incompressible noise;
		\item $\bc$ -- structured latent code $p(\bc) = \prod_{j=1}^d p(c_j)$.
	\end{itemize}
	\begin{block}{Information-theoretic regularization}
		\vspace{-0.3cm}
		\[
			\max I (\bc, G(\bz, \bc))
		\]
	\end{block}
	Information in the latent code $\bc$ should not be lost in the generation process.
	\myfootnotewithlink{https://arxiv.org/abs/1606.03657}{Chen X. et al. InfoGAN: Interpretable Representation Learning by Information Maximizing Generative Adversarial Nets, 2016}
\end{frame}
%=======
\begin{frame}{InfoGAN}
	\begin{block}{Objective}
		\vspace{-0.3cm}
		\[
		\min_{G} \max_D V(G, D) - \lambda I (\bc, G(\bz, \bc))
		\]
	\end{block}
	\begin{block}{Variational Information Maximization}
		\vspace{-0.3cm}
		\begin{multline*}
		I (\bc, G(\bz, \bc)) = H(\bc) - H(\bc | G(\bz, \bc)) = \\
		= H(\bc) + \bbE_{\bx \sim G(\bz, \bc)} \left[ \bbE_{\bc' \sim p(\bc | \bx)} \log p(\bc' | \bx) \right] = \\ 
		= H(\bc) + \bbE_{\bx \sim G(\bz, \bc)} KL(p(\bc'| \bx) || q(\bz' | \bx)) + 
		\\ + \bbE_{\bx \sim G(\bz, \bc)}  \bbE_{\bc' \sim p(\bc | \bx)} \log q(\bc' | \bx)  \geq\\
		 \geq H(\bc) + \bbE_{\bx \sim G(\bz, \bc)} \bbE_{\bc' \sim p(\bc | \bx)} \log q(\bc' | \bx) = \\
		 H(\bc) + \bbE_{\bc \sim p(\bc)} \bbE_{\bx \sim G(\bz, \bc)} \log q(\bc | \bx)
		\end{multline*}
	\end{block}
	\myfootnotewithlink{https://arxiv.org/abs/1606.03657}{Chen X. et al. InfoGAN: Interpretable Representation Learning by Information Maximizing Generative Adversarial Nets, 2016}
\end{frame}
%=======
\begin{frame}{InfoGAN}
	\begin{block}{Latent codes on MNIST}
		\begin{figure}
			\centering
			\includegraphics[width=\linewidth]{figs/infogan_mnist.png}
		\end{figure}
	\end{block}

	\myfootnotewithlink{https://arxiv.org/abs/1606.03657}{Chen X. et al. InfoGAN: Interpretable Representation Learning by Information Maximizing Generative Adversarial Nets, 2016}
\end{frame}
%=======
\begin{frame}{InfoGAN}
	\begin{block}{Latent codes on 3D Faces}
		\begin{figure}
			\centering
			\includegraphics[width=\linewidth]{figs/infogan_faces.png}
		\end{figure}
	\end{block}

	\myfootnotewithlink{https://arxiv.org/abs/1606.03657}{Chen X. et al. InfoGAN: Interpretable Representation Learning by Information Maximizing Generative Adversarial Nets, 2016}
\end{frame}
%=======
\begin{frame}{$\beta$-VAE}
	\vspace{-0.5cm}
	\begin{block}{Constrained optimization}
	\vspace{-0.5cm}
	\[
	    \max_{q, \btheta} \mathbb{E}_{q(\bz | \bx)} \log p(\bx | \bz, \btheta), \quad \text{subject to } KL (q(\bz | \bx) || p(\bz)) < \epsilon.
	\]
	\vspace{-0.7cm}
	\end{block}
	\begin{block}{Objective}
	\vspace{-0.5cm}
	\[
	    \mathcal{L}(q, \btheta, \beta) = \mathbb{E}_{q(\bz | \bx)} \log p(\bx | \bz, \btheta) - \beta \cdot KL (q(\bz | \bx) || p(\bz)).
	\]
	\vspace{-0.5cm}
	\end{block}
	What do we get at $\beta = 1$? \\
	\begin{block}{Hypothesis}
	To learn disentangled representations of the conditionally independent factors $\bv$, it is important to set a stronger constraint on the latent bottleneck: $\beta > 1$.
	\end{block}
	\textbf{Note:} It leads to poorer reconstructions and a loss of high frequency details when passing through a constrained latent bottleneck. 
	\myfootnotewithlink{https://openreview.net/references/pdf?id=Sy2fzU9gl}{Higgins I. et al. beta-VAE: Learning Basic Visual Concepts with a Constrained Variational Framework, 2017}
\end{frame}
%=======
\begin{frame}{$\beta$-VAE}
	\begin{block}{Disentangling metric}
		\begin{enumerate}
			\item Generate two sets of objects
			\[
			\bx_{li} \sim \text{Sim}(\bv_{li}, \bw_{li}); \quad \bx_{lj} \sim \text{Sim}(\bv_{lj}, \bw_{lj}); \quad y_{ij} \sim U[1, d].
			\]
			\[
			\bv_{li} \sim p(\bv); \quad \bv_{lj} \sim p(\bv) \, ([v_{li}]_y = [v_{lj}]_y); \quad \bw_{li}, \bw_{lj} \sim p(\bw).
			\]
			\item Find representations
			\[
			q(\bz | \bx) = \mathcal{N}\left(\mu(\bx) | \sigma^2(\bx)\right); \quad \bz_{li} = \mu(\bx_{li}); \quad \bz_{lj} = \mu(\bx_{lj}).
			\]
			\item Use accuracy of classifier $p(y | \bz_{\text{diff}})$ with a low VC-dimension as metric of disentanglement
			\[
			\bz_{\text{diff}} = \frac{1}{L} \sum_{l=1}^L | \bz_{li} - \bz_{lj} |.
			\]
		\end{enumerate}
	
	\end{block}
	\myfootnotewithlink{https://openreview.net/references/pdf?id=Sy2fzU9gl}{Higgins I. et al. beta-VAE: Learning Basic Visual Concepts with a Constrained Variational Framework, 2017}
\end{frame}
%=======
\begin{frame}{$\beta$-VAE}
	\begin{figure}
	    \centering
	    \includegraphics[width=0.95\linewidth]{figs/betaVAE_1.png}
	\end{figure}

	\myfootnotewithlink{https://openreview.net/references/pdf?id=Sy2fzU9gl}{Higgins I. et al. beta-VAE: Learning Basic Visual Concepts with a Constrained Variational Framework, 2017}
\end{frame}
%=======
\begin{frame}{$\beta$-VAE}
	\vspace{1cm}
	\begin{figure}
	    \centering
	    \includegraphics[width=\linewidth]{figs/betaVAE_2.png}
	\end{figure}

	\myfootnotewithlink{https://openreview.net/references/pdf?id=Sy2fzU9gl}{Higgins I. et al. beta-VAE: Learning Basic Visual Concepts with a Constrained Variational Framework, 2017}
\end{frame}
%=======
\begin{frame}{$\beta$-VAE}
	\begin{figure}
	    \centering
	    \includegraphics[width=0.8\linewidth]{figs/betaVAE_3.png}
	\end{figure}

	\myfootnotewithlink{https://openreview.net/references/pdf?id=Sy2fzU9gl}{Higgins I. et al. beta-VAE: Learning Basic Visual Concepts with a Constrained Variational Framework, 2017}
\end{frame}
%=======
\begin{frame}{$\beta$-VAE}
	\begin{minipage}[t]{0.5\columnwidth}
	    \vspace{1.5cm}
		\begin{figure}
			\centering
			\includegraphics[width=1.\linewidth]{figs/betaVAE_4.png}
		\end{figure}
	\end{minipage}%
	\begin{minipage}[t]{0.5\columnwidth}
		\begin{figure}[h]
			\centering
			\includegraphics[width=.95\linewidth]{figs/betaVAE_5.png}
		\end{figure}
	\end{minipage}

	\myfootnotewithlink{https://openreview.net/references/pdf?id=Sy2fzU9gl}{Higgins I. et al. beta-VAE: Learning Basic Visual Concepts with a Constrained Variational Framework, 2017}
\end{frame}
%=======
\begin{frame}{$\beta$-VAE}
	\begin{itemize}
		\item \textbf{Top row:} original images.
		\item \textbf{Second row:} the corresponding reconstructions. 
		\item \textbf{Remaining rows:} latent traversals ordered by KL divergence with the prior. 
		\item \textbf{Heatmaps:} latent activations for each 2D position.
	\end{itemize}
	\begin{figure}
	    \centering
	    \includegraphics[width=\linewidth]{figs/betaVAE_6.png}
	\end{figure}
	\myfootnotewithlink{https://arxiv.org/abs/1804.03599}{Burgess C. P. et al. Understanding disentangling in $\beta$-VAE, 2018}
\end{frame}
%=======
\begin{frame}{$\beta$-VAE}
	\begin{block}{Controlled encoding capacity}
	\vspace{-0.5cm}
	\[
	    \mathcal{L}(q, \btheta, \beta) = \mathbb{E}_{q(\bz | \bx)} \log p(\bx | \bz, \btheta) - | KL (q(\bz | \bx) || p(\bz)) - C|.
	\]
	\end{block}
	\begin{figure}
	    \centering
	    \includegraphics[width=0.9\linewidth]{figs/betaVAE_7.png}
	\end{figure}
	The early capacity is allocated to positional latents only, followed by a scale latent, then shape and orientation latents.

	\myfootnotewithlink{https://arxiv.org/abs/1804.03599}{Burgess C. P. et al. Understanding disentangling in $\beta$-VAE, 2018}
\end{frame}
%=======
\begin{frame}{$\beta$-VAE}
	\begin{block}{Controlled encoding capacity}
	\vspace{-0.5cm}
	\[
	    \mathcal{L}(q, \btheta, \beta) = \mathbb{E}_{q(\bz | \bx)} \log p(\bx | \bz, \btheta) - | KL (q(\bz | \bx) || p(\bz)) - C|.
	\]
	\vspace{-0.5cm}
	\end{block}
	\begin{figure}
	    \centering
	    \includegraphics[width=0.7\linewidth]{figs/betaVAE_8.png}
	\end{figure}
	As the information capacity increases the different latents associated with their data generative factors become informative.

	\myfootnotewithlink{https://arxiv.org/abs/1804.03599}{Burgess C. P. et al. Understanding disentangling in $\beta$-VAE, 2018}
\end{frame}
%=======
\begin{frame}{$\beta$-VAE}
	\begin{block}{Single latent traversals, ordered by their average KL divergence with the prior}
		\begin{figure}
		\centering
		\includegraphics[width=\linewidth]{figs/betaVAE_9.png}
	\end{figure}
	\end{block}
	\myfootnotewithlink{https://arxiv.org/abs/1804.03599}{Burgess C. P. et al. Understanding disentangling in $\beta$-VAE, 2018}
\end{frame}
%=======
\begin{frame}{$\beta$-VAE}
	\begin{block}{ELBO}
		\vspace{-0.3cm}
		\[
		\mathcal{L}(q, \btheta) = \frac{1}{n} \sum_{i=1}^n \left[ \mathbb{E}_{q(\bz_i | \bx_i)} \log p(\bx_i | \bz_i, \btheta) - \beta \cdot KL(q(\bz_i | \bx_i) || p(\bz_i)) \right].
		\]
		\vspace{-0.3cm}
	\end{block}
	\begin{block}{ELBO surgery}
		\vspace{-0.3cm}
		{\footnotesize
			\[
			\mathcal{L}(q, \btheta) = \underbrace{\frac{1}{n} \sum_{i=1}^n \mathbb{E}_{q(\bz_i | \bx_i)} \log p(\bx_i | \bz_i, \btheta)}_{\text{Reconstruction loss}} - \beta \cdot \underbrace{\mathbb{I}_{q(i, \bz)} [i, \bz]\vphantom{\sum_{i=1}}}_{\text{Mutual info}} - \beta \cdot \underbrace{KL(q(\bz) || p(\bz))\vphantom{\sum_{i=1}}}_{\text{Marginal KL}}
			\]}
	\end{block}
	\begin{block}{Minimization of MI}
	\begin{itemize}
		\item It is not necessary and not desirable for disentanglement. 
		\item It hurts reconstruction.
	\end{itemize}
	\end{block}
	\myfootnotewithlink{https://arxiv.org/abs/1804.03599}{Burgess C. P. et al. Understanding disentangling in $\beta$-VAE, 2018}
\end{frame}
%=======
\begin{frame}{DIP-VAE}
	\begin{block}{Disentangled aggregated variational posterior}
		\vspace{-0.3cm}
		\[
		q(\bz) = \bbE_{\pi(\bx)} q(\bz | \bx) = \int q(\bz | \bx) \pi(\bx) d\bx = \prod_{j=1}^d q(z_j)
		\]
		\vspace{-0.3cm}
	\end{block}
	Variational inference with disentangled prior encourages inferring factors that are close to being disentangled:
	\[
		KL(q(\bz) || \bbE_{\pi(\bx)} p(\bz | \bx)) \leq \bbE_{\pi(\bx)} KL(q(\bz | \bx) || p(\bz | \bx))	
	\]
	\begin{block}{DIP-VAE Objective}
		\vspace{-0.3cm}
		{\footnotesize
			\begin{multline*}
			\mathcal{L}(q, \btheta) = \underbrace{\bbE_{\pi(\bx)} \left[ \mathbb{E}_{q(\bz | \bx)} \log p(\bx | \bz, \btheta) - KL(q(\bz | \bx) || p(\bz)) \right]}_{\text{ELBO}} -\lambda \cdot KL(q(\bz) || p(\bz)) = \\
			= \underbrace{\bbE_{\pi(\bx)} \left[\mathbb{E}_{q(\bz | \bx)} \log p(\bx | \bz, \btheta)\right]}_{\text{Reconstruction loss}} - \underbrace{\mathbb{I}_{q(i, \bz)} [i, \bz]}_{\text{Mutual info}} - (1 + \lambda) \cdot \underbrace{KL(q(\bz) || p(\bz))}_{\text{Marginal KL}}
			\end{multline*}
		}
		\vspace{-0.3cm}
	\end{block}

	\myfootnotewithlink{https://arxiv.org/abs/1711.00848}{Kumar A., Sattigeri P., Balakrishnan A. Variational Inference of Disentangled Latent Concepts from Unlabeled Observations, 2017}
\end{frame}
%=======
\begin{frame}{DIP-VAE}
	\begin{block}{DIP-VAE Objective}
		\vspace{-0.3cm}
		{\footnotesize
			\[
				\mathcal{L}(q, \btheta) = \underbrace{\bbE_{\pi(\bx)} \left[ \mathbb{E}_{q(\bz | \bx)} \log p(\bx | \bz, \btheta) - KL(q(\bz | \bx) || p(\bz)) \right]}_{\text{ELBO}} -\lambda \cdot KL(q(\bz) || p(\bz))
			\]
		}
		\vspace{-0.3cm}
	\end{block}
	\begin{itemize}
		\item $KL(q(\bz) || p(\bz))$ is intractable.
		\item Let match the moments of $q(\bz)$ and $p(\bz)$.
		\[
		\text{cov}_{q(\bz)}(\bz) = \bbE_{\pi(\bx)} \text{cov}_{q(\bz|\bx)}(\bz) + \text{cov}_{\pi(\bx)} \left( \bbE_{q(\bz | \bx)}\bz \right).
		\]
		For most common case $q(\bz | \bx) = \cN(\bmu(\bx), \bSigma(\bx))$:
		\[
		\text{cov}_{q(\bz)}(\bz) = \bbE_{\pi(\bx)} \bSigma(\bx) + \text{cov}_{\pi(\bx)} \bmu(\bx)
		\]
		DIP-VAE regularizes $\text{cov}_{q(\bz)}(\bz) $ to be close to the identity matrix.
	\end{itemize}

	\myfootnotewithlink{https://arxiv.org/abs/1711.00848}{Kumar A., Sattigeri P., Balakrishnan A. Variational Inference of Disentangled Latent Concepts from Unlabeled Observations, 2017}
\end{frame}
%=======
\begin{frame}{DIP-VAE}
	\begin{block}{DIP-VAE Objective}
		\vspace{-0.3cm}
		{\footnotesize
			\[
			\mathcal{L}(q, \btheta) = \underbrace{\bbE_{\pi(\bx)} \left[ \mathbb{E}_{q(\bz | \bx)} \log p(\bx | \bz, \btheta) - KL(q(\bz | \bx) || p(\bz)) \right]}_{\text{ELBO}} -\lambda \cdot KL(q(\bz) || p(\bz))
			\]
		}
		\vspace{-0.5cm}
	\end{block}
	\[
		\text{cov}_{q(\bz)}(\bz) = \bbE_{\pi(\bx)} \bSigma(\bx) + \text{cov}_{\pi(\bx)} \bmu(\bx)
	\]
	\vspace{-0.3cm}
	\begin{block}{DIP-VAE-I}
		\vspace{-0.3cm}
		\[
			\max_{\btheta, \bphi} \text{ELBO}(\btheta, \bphi) - \lambda_1 \sum_{i \neq j} \left[\text{cov}_{\pi(\bx)} \bmu(\bx)\right]^2_{ij} - \lambda_2 \sum_{i} \left( \left[ \text{cov}_{\pi(\bx)} \bmu(\bx) \right]_{ii} - 1 \right)^2
		\]
		\vspace{-0.3cm}
	\end{block}
	\begin{block}{DIP-VAE-II}
		\vspace{-0.3cm}
		\[
			\max_{\btheta, \bphi} \text{ELBO}(\btheta, \bphi) - \lambda_1 \sum_{i \neq j} \left[\text{cov}_{q(\bz)} (\bz) \right]^2_{ij} - \lambda_2 \sum_{i} \left( \left[ \text{cov}_{q(\bz)} (\bz) \right]_{ii} - 1 \right)^2
		\]
		\vspace{-0.3cm}
	\end{block}

	\myfootnotewithlink{https://arxiv.org/abs/1711.00848}{Kumar A., Sattigeri P., Balakrishnan A. Variational Inference of Disentangled Latent Concepts from Unlabeled Observations, 2017}
\end{frame}
%=======
\begin{frame}{DIP-VAE}
	\begin{figure}
		\centering
		\includegraphics[width=0.8\linewidth]{figs/dip-vae_1}
	\end{figure}
	\vfill
	\hrule\medskip
	{\scriptsize \href{https://arxiv.org/abs/1711.00848}{https://arxiv.org/abs/1711.00848}}
\end{frame}
%=======
\begin{frame}{DIP-VAE}
	\begin{figure}
		\centering
		\includegraphics[width=0.95\linewidth]{figs/dip-vae_2}
	\end{figure}

	\myfootnotewithlink{https://arxiv.org/abs/1711.00848}{Kumar A., Sattigeri P., Balakrishnan A. Variational Inference of Disentangled Latent Concepts from Unlabeled Observations, 2017}
\end{frame}
%=======
\begin{frame}{FactorVAE}
	\begin{block}{Disentangled aggregated variational posterior}
		\vspace{-0.3cm}
		\[
		q(\bz) = \bbE_{\pi(\bx)} q(\bz | \bx) = \int q(\bz | \bx) \pi(\bx) d\bx = \prod_{j=1}^d q(z_j)
		\]
		\vspace{-0.3cm}
	\end{block}
	\begin{block}{Total correlation regularizer}
		\vspace{-0.3cm}
		\[
		\min KL(q(\bz) || \prod_{j=1}^d q(z_j))
		\]
		\vspace{-0.3cm}
	\end{block}
	\begin{block}{FactorVAE objective}
		\vspace{-0.3cm}
		\[
		\min_{\btheta, \bphi} \text{ELBO}(\btheta, \bphi) - \gamma \cdot KL(q(\bz) || \prod_{j=1}^d q(z_j))
		\]
		\vspace{-0.3cm}
	\end{block}
	\begin{itemize}
		\item The last term is intractable.
		\item FactorVAE uses density ratio trick for estimation. 
	\end{itemize}

	\myfootnotewithlink{https://arxiv.org/abs/1802.05983}{Kim H., Mnih A. Disentangling by Factorising, 2018}
\end{frame}
%=======
\begin{frame}{FactorVAE}
	Consider two distributions $q_1(\bx)$, $q_2(\bx)$ and probilistic model
	\[
		p(\bx | y) = \begin{cases}
			q_1(\bx), \text{ if } y = 1, \\
			q_2(\bx), \text{ if } y = 0,
		\end{cases}
		\quad 
		y \sim \text{Bern}(0.5).
	\]
	\begin{block}{Density ratio trick}
		\vspace{-0.5cm}
		\begin{multline*}
			\frac{q_1(\bx)}{q_2(\bx)} = \frac{p(\bx | y = 1)}{p(\bx | y = 0)} = \frac{p(y = 1 | \bx) p(\bx)}{p(y=1)} \bigg/ \frac{p(y = 0 | \bx) p(\bx)}{p(y=0)} = \\
			= \frac{p(y = 1 | \bx)}{p(y = 0 | \bx)} = \frac{p(y = 1 | \bx)}{1 - p(y = 1 | \bx)} = \frac{D(\bx)}{1 - D(\bx)}
		\end{multline*}
	Here $D(\bx)$ could be treated as a discriminator a model the output of which is a probability that $\bx$ is a sample
	from $q_1(\bx)$ rather than from $q_2(\bx)$.
	\end{block}

	\myfootnotewithlink{https://arxiv.org/abs/1802.05983}{Kim H., Mnih A. Disentangling by Factorising, 2018}
\end{frame}
%=======
\begin{frame}{FactorVAE}
	
	\begin{block}{FactorVAE objective}
		\vspace{-0.3cm}
		\[
		\min_{\btheta, \bphi} \text{ELBO}(\btheta, \bphi) - \gamma \cdot KL(q(\bz) || \prod_{j=1}^d q(z_j))
		\]
		\vspace{-0.3cm}
	\end{block}
	
	\begin{block}{Total correlation regularizer}
		\vspace{-0.7cm}
		\begin{multline*}
		KL(q(\bz) || \prod_{j=1}^d q(z_j)) = KL(q(\bz) || \bar{q}(\bz)) = \\ =\bbE_{q(\bz)} \log \frac{q(\bz)}{\bar{q}(\bz)} \approx \bbE_{q(\bz)} \log \frac{D(\bz)}{1 - D(\bz)}
		\end{multline*}
		\vspace{-0.3cm}
	\end{block}
	VAE and GAN are trained simultaneously.

	\myfootnotewithlink{https://arxiv.org/abs/1802.05983}{Kim H., Mnih A. Disentangling by Factorising, 2018}
\end{frame}
%=======
\begin{frame}{FactorVAE}
	
	\begin{minipage}[t]{0.5\columnwidth}
		\begin{figure}
			\centering
			\includegraphics[width=\linewidth]{figs/factorvae_1}
		\end{figure}
	\end{minipage}%
	\begin{minipage}[t]{0.5\columnwidth}
		\vspace{1.7cm}
		\begin{figure}[h]
			\centering
			\includegraphics[width=\linewidth]{figs/factorvae_2}
		\end{figure}
	\end{minipage}

	\myfootnotewithlink{https://arxiv.org/abs/1802.05983}{Kim H., Mnih A. Disentangling by Factorising, 2018}
\end{frame}
%=======
\begin{frame}{Challenging Disentanglement Assumptions}
	Whether unsupervised disentanglement learning is even possible for arbitrary generative models?
	
	\begin{block}{Theorem}
		For $d > 1$, let $\bz \sim P$ denote any distribution which admits a density $p(\bz) = \prod^d_{i=1} p(z_i)$. Then, there exists an infinite family of bijective functions $f : \text{supp}(\bz) \rightarrow \text{supp}(\bz)$ such that
		\begin{itemize}
			\item $\frac{\partial f_i(\bu)}{\partial u_j} \neq 0$ almost everywhere for all $i$ and $j$ (i.e., $\bz$ and $f(\bz)$ are completely entangled);
			\item and $P(\bz \leq \bu) = P(f(\bz) \leq \bu)$ for all $\bu \in \text{supp}(\bz)$ (i.e., they
			have the same marginal distribution).
		\end{itemize}  
	\end{block}

	Theorem claims that unsupervised disentanglement learning is impossible for arbitrary generative models with a factorized prior.
	
	\myfootnotewithlink{https://arxiv.org/abs/1811.12359}{Locatello F. et al. Challenging Common Assumptions in the Unsupervised Learning of Disentangled Representations, 2018}
\end{frame}
%=======
\begin{frame}{Challenging Disentanglement Assumptions}
	Assume we have $p(\bz)$ and some $p(\bx|\bz)$ defining a generative model. Consider any unsupervised
	disentanglement method and assume that it finds a representation that is perfectly disentangled with respect
	to $\bz$ in the generative model.
	\begin{itemize}
		\item Theorem claims that $\exists$ $\hat{\bz} = f(\bz)$ where $\hat{\bz}$ is completely entangled
		with respect to $\bz$.
		\item Since the (unsupervised) disentanglement method only has access to
		observations $\bx$, it hence cannot distinguish between the two equivalent generative models and thus has to be entangled to at least one of them
		\[
			p(\bx) = \int p(\bx | \bz) p(\bz) d\bz = \int p(\bx | \hat{\bz})p(\hat{\bz}) d \hat{\bz}.
		\]
	\end{itemize}

	\myfootnotewithlink{https://arxiv.org/abs/1811.12359}{Locatello F. et al. Challenging Common Assumptions in the Unsupervised Learning of Disentangled Representations, 2018}
\end{frame}
%=======
\begin{frame}{Challenging Disentanglement Assumptions}
	\begin{block}{Proof (1)}
		\begin{enumerate}
			\item 
			Consider the function $g: \text{supp}(\bz) \rightarrow [0, 1]^d$:
			\vspace{-0.1cm}
			\[
				g_i(\bv) = P(z_i \leq v_i), \quad i=1, \dots, d.
			\]
			\vspace{-0.4cm}
			\begin{itemize}
				\item $g$ is bijective (since $p(\bz) = \prod_{i=1}dp(z_i)$).
				\item $\frac{\partial g_i(\bu)}{\partial u_i} \neq 0$, for all $i$ and $\frac{\partial g_i(\bu)}{\partial u_j} = 0$ for all $i \neq j$.
				\item $g(\bz)$ is an independent $d$-dimensional uniform distribution.
			\end{itemize}
			\item 
			Consider $h: (0, 1]^d \rightarrow \bbR^d$
			\[
				h_i(\bv) = \psi^{-1}(v_i), \quad i= 1, \dots, d.
			\]
			Here $\psi$  denotes the CDF of a standard normal distribution.
			\begin{itemize}
				\item $h$ is bijective.
				\item $\frac{\partial g_i(\bu)}{\partial u_i} \neq 0$, for all $i$ and $\frac{\partial g_i(\bu)}{\partial u_j} = 0$ for all $i \neq j$.
				\item $h(g(\bz))$  is a $d$-dimensional standard normal distribution.
			\end{itemize}
		\end{enumerate}
	\end{block}

	\myfootnotewithlink{https://arxiv.org/abs/1811.12359}{Locatello F. et al. Challenging Common Assumptions in the Unsupervised Learning of Disentangled Representations, 2018}
\end{frame}
\begin{frame}{Challenging Disentanglement Assumptions}
	\begin{block}{Proof (2)}
		Let $\bA \in \bbR^{d \times d}$ be an arbitrary orthogonal matrix with $A_{ij} \neq 0$ for all $i, j$.
		The family of such matrices is infinite.
		\begin{itemize}
			\item $\bA$ is orthogonal, it is invertible and thus defines a bijective linear operator. 
			\item $\bA h(g(\bz)) \in \bbR^d$ is hence an independent, multivariate standard normal distribution.
			\item $h^{-1}( \bA h(g(\bz))) \in \bbR^d$ is an independent $d$-dimensional uniform distribution.
		\end{itemize}
		Define $f: \text{supp}(\bz) \rightarrow \text{supp}(\bz)$:
		\[
			f(\bu) = g^{-1} (h^{-1}( \bA h(g(\bz)))).
		\]
		By definition $f(\bz)$ has the same marginal distribution as $\bz$: $P(\bz \leq \bu) = P(f(\bz) \leq \bu)$ and $\frac{\partial f_i(\bu)}{\partial u_j} \neq 0$.
		\end{block}

	\myfootnotewithlink{https://arxiv.org/abs/1811.12359}{Locatello F. et al. Challenging Common Assumptions in the Unsupervised Learning of Disentangled Representations, 2018}
\end{frame}
%=======
\begin{frame}{Challenging Disentanglement Assumptions}
	\begin{itemize}
		\item \textbf{Training:} Factorizing \textbf{samples} from aggregated posterior $q(\bz) = \prod_{i=1}^d q(z_i)$.
		\item \textbf{Inference:} Use a \textbf{mean} vector (usually mean of Gaussian encoder) as a representation.
	\end{itemize}
	\begin{figure}
		\centering
		\includegraphics[width=0.95\linewidth]{figs/challenge_dis_1}
	\end{figure}

	\myfootnotewithlink{https://arxiv.org/abs/1811.12359}{Locatello F. et al. Challenging Common Assumptions in the Unsupervised Learning of Disentangled Representations, 2018}
\end{frame}
%=======
\begin{frame}{Challenging Disentanglement Assumptions}
	\begin{block}{Importance of different models and hyperparameters for disentanglement}
		\begin{figure}
			\centering
			\includegraphics[width=\linewidth]{figs/challenge_dis_2}
		\end{figure}
	\end{block}

	\myfootnotewithlink{https://arxiv.org/abs/1811.12359}{Locatello F. et al. Challenging Common Assumptions in the Unsupervised Learning of Disentangled Representations, 2018}
\end{frame}
%=======
\begin{frame}{Challenging Disentanglement Assumptions}
	\begin{block}{Agreement of different disentanglement metrics}
		\begin{figure}
			\centering
			\includegraphics[width=0.9\linewidth]{figs/challenge_dis_3}
		\end{figure}
		\vspace{0.5cm}
	\end{block}

	\myfootnotewithlink{https://arxiv.org/abs/1811.12359}{Locatello F. et al. Challenging Common Assumptions in the Unsupervised Learning of Disentangled Representations, 2018}
\end{frame}
%=======
\begin{frame}{Summary}
\end{frame}
\end{document} 