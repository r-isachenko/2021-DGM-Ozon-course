\documentclass{article}
\usepackage[utf8]{inputenc}
\usepackage[T1]{fontenc}
\usepackage[russian]{babel}

\usepackage{graphicx, epsfig}
\usepackage{amsmath,mathrsfs,amsfonts,amssymb}

\usepackage{geometry} % Простой способ задавать поля
	\geometry{top=25mm}
	\geometry{bottom=25mm}
	\geometry{left=25mm}
	\geometry{right=20mm}

% latin bold lower
\newcommand{\ba}{\mathbf{a}} 
\newcommand{\bc}{\mathbf{c}} 
\newcommand{\be}{\mathbf{e}} 
\newcommand{\bh}{\mathbf{h}} 
\newcommand{\bp}{\mathbf{p}} 
\newcommand{\bt}{\mathbf{t}} 
\newcommand{\bs}{\mathbf{s}} 
\newcommand{\bu}{\mathbf{u}} 
\newcommand{\bv}{\mathbf{v}} 
\newcommand{\bw}{\mathbf{w}} 
\newcommand{\bx}{\mathbf{x}} 
\newcommand{\by}{\mathbf{y}} 
\newcommand{\bz}{\mathbf{z}} 

% latin bold upper
\newcommand{\bA}{\mathbf{A}} 
\newcommand{\bB}{\mathbf{B}} 
\newcommand{\bC}{\mathbf{C}} 
\newcommand{\bI}{\mathbf{I}} 
\newcommand{\bM}{\mathbf{M}} 
\newcommand{\bT}{\mathbf{T}} 
\newcommand{\bU}{\mathbf{U}} 
\newcommand{\bV}{\mathbf{V}} 
\newcommand{\bW}{\mathbf{W}} 
\newcommand{\bX}{\mathbf{X}} 
\newcommand{\bY}{\mathbf{Y}} 
\newcommand{\bZ}{\mathbf{Z}} 

% latin cal upper
\newcommand{\cL}{\mathcal{L}} 
\newcommand{\cN}{\mathcal{N}} 
\newcommand{\cS}{\mathcal{S}} 
\newcommand{\cT}{\mathcal{T}} 
\newcommand{\cW}{\mathcal{W}} 
\newcommand{\cX}{\mathcal{X}} 
\newcommand{\cZ}{\mathcal{Z}} 

% latin cal upper
\newcommand{\bbE}{\mathbb{E}} 
\newcommand{\bbP}{\mathbb{P}} 
\newcommand{\bbR}{\mathbb{R}} 

% greek bold lower
\newcommand{\bepsilon}{\boldsymbol{\epsilon}} 
\newcommand{\btheta}{\boldsymbol{\theta}} 
\newcommand{\blambda}{\boldsymbol{\lambda}} 
\newcommand{\bpi}{\boldsymbol{\pi}} 
\newcommand{\bmu}{\boldsymbol{\mu}} 
\newcommand{\bsigma}{\boldsymbol{\sigma}} 
\newcommand{\bphi}{\boldsymbol{\phi}} 

% greek bold upper
\newcommand{\bSigma}{\boldsymbol{\Sigma}} 

\DeclareMathOperator*{\argmin}{arg\,min}
\DeclareMathOperator*{\argmax}{arg\,max}

\begin{document}
\begin{center}
    {\Large \textbf{Homework 2}} \\
\end{center}

{\large \textbf{Autoregressive models}}
\begin{enumerate}
    \item Рассмотрим модель MADE с одним скрытым слоем $\bx \in \bbR^m$. Обозначим за $\bW \in \bbR^{h \times m}$ матрицу весов между входным и скрытым слоем, а за $\bV \in \bbR^{m \times h}$ матрицу весов между скрытым и выходным слоем ($h$ -- число нейронов в скрытом слое). Пусть мы сгенерировали корректные авторегрессионные маски $\bM_{\bW} \in \bbR^{h \times m}$ и $\bM_{\bV} \in \bbR^{m \times h}$ (алгоритм генерации приведен в лекции 1) для прямого порядка переменных ($p(\bx) = p(x_1) \cdot p(x_2 | x_1) \cdot \dots \cdot p(x_m | x_{m-1}, \dots, x_1)$). Каждая маска является бинарной матрицей из 0 и 1. Введем матрицу $\bM = \bM_{\bV} \bM_{\bW}$. Докажите, что:
    \begin{enumerate}
    	\item  \textbf{(2 pt)} $\bM$ строго нижняя треугольная (имеет нули на диагонали и выше диагонали);
    	\item  \textbf{(2 pt)} $\bM_{ij}$ равно числу путей в графе сети между выходным нейроном $\hat{x}_i$ и входным нейроном $x_j$.
    \end{enumerate}
    \item Пусть у нас есть 2 генеративные модели для изображений размера $W \times H \times C$, где $W$ - ширина изображения, $H$ - высота, $C$ - число каналов. Первая модель $p_1(\bx | \btheta)$ выдает дискретное распределение для каждого пикселя $\text{Categorical}(\bpi)$, где $\bpi = (\pi_1, \dots,  \pi_{256})$. Вторая модель $p_2(\bx | \btheta)$ моделирует дискретное распределение непрерывной смесью логистических функций
   		\[
  				p(\nu | \bmu, \bs, \bpi) = \sum_{i=1}^K \pi_k p(\nu | \mu_k, s_k).
		\]
    	\[
    		P(x | \bmu, \bs, \bpi) = P(x + 0.5 | \bmu, \bs, \bpi) - P(x - 0.5 | \bmu, \bs, \bpi).
    	\]
    Каждая из моделей выдает параметры распределений пикселей.
    	\begin{enumerate}
    		\item \textbf{(1 pt)} Посчитайте размерность выходного тензора для модели $p_1(\bx | \btheta)$ и для модели $p_2(\bx | \btheta)$. 
    		\item \textbf{(1 pt)} При каком числе компонент смеси $K$ число элементов выходного тензора для $p_2(\bx | \btheta)$ становится больше, чем для $p_1(\bx | \btheta)$.
    	\end{enumerate}
\end{enumerate}

{\large \textbf{Latent Variable models}}
\begin{enumerate}
	\item \textbf{(2 pt)} Пусть имеется два распределения $p_1(\bx | \bmu_1, \bSigma_1) = \cN(\bmu_1, \bSigma_1)$, $p_2(\bx | \bmu_2, \bSigma_2) = \cN(\bmu_2, \bSigma_2)$. Выведите формулу для $KL(p_1 || p_2)$.
	\item На лекции 3 при выводе градиента ELBO на E-шаге мы столкнулись с проблемой при Монте-Карло оценивании, так как функция распределения зависела от параметров дифференцирования.
		\begin{align*}
			\nabla_{\bphi} \mathcal{L} (\bphi, \btheta) &= \nabla_{\bphi} \int q(\bz | \bx, \bphi) \left[\log p(\bx, \bz | \btheta) - \log q(\bz| \bx, \bphi) \right] d \bz \\
			& \neq  \int q(\bz | \bx, \bphi) \nabla_{\bphi} \left[\log p(\bx, \bz | \btheta) - \log q(\bz| \bx, \bphi) \right] d \bz \\
		\end{align*}
	 Reparametrization trick позволил пробросить градиент и получить Монте-Карло оценку. Но есть и другой способ, который использует log-derivative trick:
	 \[
	 	  \nabla_\xi  \log q(\eta| \xi) = \frac{\nabla_\xi q(\eta| \xi)}{q(\eta| \xi)}.
	 \] 
	 \begin{enumerate}
	 \item \textbf{(2 pt)} Используя формулу для производной логарифма получите Монте-Карло оценку градиента.
	 \item \textbf{(2 pt)} Полученная оценка работает существенно хуже, чем reparametrization trick. А именно обладает огромной дисперсий. Попробуйте описать интуицию, почему оценка обладает высоким разбросом (для этого нужно подумать какого порядка и знака будут иметь члены, участвующие в оценке).
	 \end{enumerate}
	 
	  \item \textbf{(3 pt)} В курсе нам встретятся дивергенции, отличные от $KL$. Поэтому давайте познакомимся с целым классом $\alpha$-дивергенций:
	  \[
	 	  	D_{\alpha}(p || q) = \frac{4}{1 - \alpha^2} \left( 1 - \int p(x)^{\frac{1 + \alpha}{2}}q(x)^{\frac{1 - \alpha}{2}}dx\right).
	  \]
	  Для любого значения $\alpha \in [-\infty; +\infty]$ функция $D_{\alpha} (p || q)$ будет задавать некоторую меру схожести двух распределений, обладающую свои свойствами.
	  
	  Докажите, что при $\alpha \rightarrow 1$ дивергенция $D_{\alpha}(p || q) \rightarrow KL(p || q)$, а при $\alpha \rightarrow -1$ дивергенция $D_{\alpha}(p || q) \rightarrow KL(q || p)$. При доказательстве используйте факт, что $t^\epsilon = \exp(\epsilon \ln t) = 1 + \epsilon \ln t + O(\epsilon^2)$.
\end{enumerate}

\end{document}
