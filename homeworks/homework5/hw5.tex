\documentclass{article}
\usepackage[utf8]{inputenc}
\usepackage[T1]{fontenc}
\usepackage[russian]{babel}

\usepackage{graphicx, epsfig}
\usepackage{amsmath,mathrsfs,amsfonts,amssymb}

\usepackage{geometry} % Простой способ задавать поля
	\geometry{top=25mm}
	\geometry{bottom=25mm}
	\geometry{left=25mm}
	\geometry{right=20mm}

% latin bold lower
\newcommand{\ba}{\mathbf{a}} 
\newcommand{\bc}{\mathbf{c}} 
\newcommand{\be}{\mathbf{e}} 
\newcommand{\bh}{\mathbf{h}} 
\newcommand{\bp}{\mathbf{p}} 
\newcommand{\bt}{\mathbf{t}} 
\newcommand{\bs}{\mathbf{s}} 
\newcommand{\bu}{\mathbf{u}} 
\newcommand{\bv}{\mathbf{v}} 
\newcommand{\bw}{\mathbf{w}} 
\newcommand{\bx}{\mathbf{x}} 
\newcommand{\by}{\mathbf{y}} 
\newcommand{\bz}{\mathbf{z}} 

% latin bold upper
\newcommand{\bA}{\mathbf{A}} 
\newcommand{\bB}{\mathbf{B}} 
\newcommand{\bC}{\mathbf{C}} 
\newcommand{\bI}{\mathbf{I}} 
\newcommand{\bM}{\mathbf{M}} 
\newcommand{\bT}{\mathbf{T}} 
\newcommand{\bU}{\mathbf{U}} 
\newcommand{\bV}{\mathbf{V}} 
\newcommand{\bW}{\mathbf{W}} 
\newcommand{\bX}{\mathbf{X}} 
\newcommand{\bY}{\mathbf{Y}} 
\newcommand{\bZ}{\mathbf{Z}} 

% latin cal upper
\newcommand{\cL}{\mathcal{L}} 
\newcommand{\cN}{\mathcal{N}} 
\newcommand{\cS}{\mathcal{S}} 
\newcommand{\cT}{\mathcal{T}} 
\newcommand{\cW}{\mathcal{W}} 
\newcommand{\cX}{\mathcal{X}} 
\newcommand{\cZ}{\mathcal{Z}} 

% latin cal upper
\newcommand{\bbE}{\mathbb{E}} 
\newcommand{\bbP}{\mathbb{P}} 
\newcommand{\bbR}{\mathbb{R}} 

% greek bold lower
\newcommand{\bepsilon}{\boldsymbol{\epsilon}} 
\newcommand{\btheta}{\boldsymbol{\theta}} 
\newcommand{\blambda}{\boldsymbol{\lambda}} 
\newcommand{\bpi}{\boldsymbol{\pi}} 
\newcommand{\bmu}{\boldsymbol{\mu}} 
\newcommand{\bsigma}{\boldsymbol{\sigma}} 
\newcommand{\bphi}{\boldsymbol{\phi}} 

% greek bold upper
\newcommand{\bSigma}{\boldsymbol{\Sigma}} 

\DeclareMathOperator*{\argmin}{arg\,min}
\DeclareMathOperator*{\argmax}{arg\,max}

\begin{document}
\begin{center}
    {\Large \textbf{Homework 5}} \\
\end{center}

\begin{enumerate}
    \item На лекции мы с вами изучили планарные потоки вида:
    \begin{equation}
    	\bx = g(\bz, \btheta) = \bz + \bu h(\bw^T\bz + b),
    \end{equation}
    где $\bu \in \bbR^m$,  $\bw \in \bbR^m$, $b \in \bbR$.
    Существует естественное обобщение планарных потоков вида:
    \begin{equation}
    	\bx = g(\bz, \btheta) = \bz + \bV h(\bW^T\bz + \mathbf{b}),
    	\label{sylvester}
    \end{equation}
    где $\bV \in \bbR^{m \times k}$,  $\bW \in \bbR^{m \times k}$, $\mathbf{b} \in \bbR^k$. Такой поток получил название \textit{Sylvester flow}.
    \begin{enumerate}
    	\item  \textbf{(1 pt)} Докажите упрощенный вариант matrix-determinant lemma:
    	\[
    		\det (\bI_m + \bV \bW^T) = \det (\bI_k + \bW^T \bV).
    	\]
    	\item  \textbf{(1.5 pt)} Посчитайте детерминант матрицы Якоби для преобразования~\eqref{sylvester} и примените к нему доказанную в предыдущем пункте лемму.
    	\item  \textbf{(1 pt)} Для того чтобы уменьшить сложность подсчета полученного детерминанта авторы предложили применить QR-разложение вида:
    	\[
    		\bV = \bQ \bU; \quad \bW = \bQ \bL,
    	\]
    	где $\bQ \in \bbR^{m \times k}$ -- ортогональная матрица ($\bQ^T \bQ = \bI$), $\bU \in \bbR^{k \times k}$ -- верхнетреугольная матрица, $\bL \in \bbR^{k \times k}$ -- нижнетреугольная матрица. Выпишите выражение для детерминанта якобиана, используя это разложение. 
    	\item  \textbf{(1.5 pt)} Вычислите и сравните сложности для подсчета детерминанта матрицы Якоби до применения леммы, после применения леммы и после применения QR-разложения
    \end{enumerate}
    
    \item  \textbf{(2 pt)} На лекции мы с вами доказали теорему об \textit{операции над ELBO}:
    \[
    	\frac{1}{n} \sum_{i=1}^n KL(q(\bz | \bx_i) || p(\bz)) = KL(q(\bz) || p(\bz)) + \bbI_{q} [\bx, \bz],
    \]
    где первый член $KL(q(\bz) || p(\bz))$ включает в себя агрегированное апостериорное распределение $q(\bz)$ и априорное распределение $p(\bz)$. Наша цель сейчас -- разобраться со вторым членом. На лекции второй член был равен:

    \begin{equation}
    	\bbI_{q} [\bx, \bz] = \frac{1}{n}\sum_{i=1}^n KL(q(\bz | \bx_i) || q (\bz)).
	   	 \label{mut_info1}
    \end{equation}
    На самом деле это совместная информация между $\bx$ и $\bz$ по эмпирическому распределению данных и распределению $q(\bz | \bx)$. Представим индекс объекта выборки $i$ - случайной величиной.
    
    \[
        q(i, \bz) = q(i) q(\bz | i); \quad p(i, \bz) = p(i) p(\bz); \quad 
        q(i) = p(i) = \frac{1}{n}.
    \]
    \[
    	 \quad q(\bz | i) = q(\bz | \bx_i) \quad q(\bz) = \sum_{i=1}^n q(i, \bz) = \frac{1}{n} \sum_{i=1}^n q(\bz | \bx_i); \quad  
    \]
    Совместная информация - это некая мера независимости двух случайных величин:
    \begin{equation}
	   	 \mathbb{I}_{q} [\bx, \bz] = \mathbb{E}_{q(i, \bz)} \log \frac{q(i, \bz)}{q(i)q(\bz)}.
	   	 \label{mut_info2}
    \end{equation}
    Докажите, что выражение~\eqref{mut_info2} равно выражению~\eqref{mut_info1}.
    
    \item  \textbf{(2 pt)} Вспомним вариационную нижнюю оценку для задачи деквантизации дискретных данных:
    \[
     	\log P(\bx | \btheta) \geq  \int q(\bu | \bx) \log \frac{p(\bx + \bu | \btheta)}{q(\bu | \bx)} d \bu = \mathcal{L}(q, \btheta).
    \]
    Мы с вами обсуждали, что вариационную нижнюю оценку можно улучшить с помощью Importance Sampling. Выпишите нижнюю оценку для деквантизации, используя $\cL_k$ по аналогии с моделью IWAE (необходимо использовать не одну $\bu$, а набор из $\{\bu_k\}_{k=1}^K$).
    
   	\item Стандартный GAN часто страдает от проблем с затухающим градиентом. Least Squares GAN пытается решить данную проблему заменой функции ошибки на следующую:
   	\[
   		\min_D V(D) = \min_D \frac{1}{2}\left[ \bbE_{\pi(\bx)} (D(\bx) - b)^2 + \bbE_{p(\bz)} (D(G(\bz)) - a)^2 \right]
   	\]
   	\[
   		\min_G V(G) = \min_G \frac{1}{2}\left[ \bbE_{\pi(\bx)} (D(\bx) - c)^2 + \bbE_{p(\bz)} (D(G(\bz)) - c)^2 \right],
   	\]
   	где $a,b,c \in \bbR$ -- фиксированные константы.
   	\begin{enumerate}
   		\item  \textbf{(1.5 pt)} Найдите формулу для оптимального дискриминатора $D^*$.
   		\item  \textbf{(1 pt)} Выпишите выражение функции ошибок гененатора $V(G)$ в случае оптимального дискриминатора $D^*$.
   		\item  \textbf{(1.5 pt)} Докажите, что при $b - c = 1$, $b - a = 2$, функция ошибок генератора $V(G)$ в случае оптимального дискриминатора $D^*$ принимает вид:
   		\[
   			V(G) = \frac{1}{2} \chi^2_{\text{Pearson}} (\pi(\bx) + p(\bx | \btheta) || 2 p(\bx | \btheta)), 
   		\]
   		где $\chi^2_{\text{Pearson}} (p || q)$ - квадратичная дивергенция Пирсона:
   		\[
   			\chi^2_{\text{Pearson}} (p || q) = \int \frac{(p(\bx) - q(\bx))^2}{p(\bx)} d \bx.
   		\]
   	\end{enumerate}
\end{enumerate}

\end{document}
